\documentclass[11pt]{article}

% Load CJE style
\usepackage{style/cje}

% Version command (auto-generated by Makefile)
\input{VERSION}

% Header/Footer setup
\pagestyle{fancy}
\fancyhf{}
\fancyhead[L]{\textit{CJE Practitioners' Playbook}}
\fancyhead[R]{\textit{v\cjeversion}}
\fancyfoot[C]{\thepage}
\renewcommand{\headrulewidth}{0.4pt}
\renewcommand{\footrulewidth}{0.4pt}

% Title and author
\title{%
  Causal Judge Evaluation (CJE): \\
  A Practitioners' Playbook for Reliable LLM-as-Judge Evaluation \\
  \large Version \cjeversion
}
\author{Eddie Landesberg \\ \texttt{eddie@cimo-labs.com}}
\date{\today}

\begin{document}

\maketitle

\begin{abstract}
LLM-as-judge has become the fastest way to compare model policies, yet most deployments still treat raw judge scores as heuristics: averages are taken at face value, failure modes are opaque, and confidence intervals are absent. This paper presents \textbf{Causal Judge Evaluation (CJE)}, a practitioner's playbook that turns judge scores into causally interpretable estimates and reliable rankings with honest uncertainty.

CJE centers evaluation on a single question---\emph{what would the KPI be if we deployed policy $\pi$?}---and provides three complementary analysis modes: \textbf{Direct Modeling (DM)} for on-policy evaluation on a shared prompt set; \textbf{Calibrated IPS} for reusing a single judged log across many candidates; and \textbf{Calibrated DR} that combines a critic with stabilized weights for robustness and tighter intervals.

Two light-weight components make the system practical. \textbf{\autocal} learns a mean-preserving, largely monotone mapping from judge score to outcome on a small labeled ``oracle'' slice, so estimates are on the right scale. \textbf{\oua} (oracle-uncertainty aware) inference then propagates calibration noise into confidence intervals. For off-policy reuse, \textbf{\simcal} stabilizes importance weights with a mean-one, score-indexed projection that raises effective sample size and tames tails without changing the target.

To keep operators in control, CJE surfaces a short list of high-leverage diagnostics---score coverage, calibration reliability, ESS and tail heaviness for IPS/DR, and a DR orthogonality check---each paired with concrete fixes (e.g., targeted labeling, cohort restriction, stronger critics). The paper spells out the assumptions required for DM, IPS, and DR to be read causally, offers copy-and-run recipes, and ships minimal artifacts for audit and reproducibility.
\end{abstract}

\tableofcontents
\newpage

% Include sections
\section{Introduction}

Large language models (LLMs) are increasingly used as judges: a model (or rubricized prompt) assigns a scalar score $S = s(X, A)$ to an answer $A$ given a prompt $X$. This setup is attractive: it is fast, cheap, and often strongly correlated with human preferences. As a result, many teams now compare policies by averaging raw judge scores and declaring a winner.

\subsection{What practitioners really want}

In day-to-day evaluation work there are two goals: (i) rank candidate policies on a shared prompt set; and (ii) when stakes justify it, estimate levels for a KPI that matters (e.g., pass rate), with a confidence interval. Both questions are causal: \emph{what would happen if we shipped policy $\pi$?} Formally, the target is
\begin{equation}
V(\pi) = \E[Y(\pi)],
\end{equation}
where $Y \in [0, 1]$ is the outcome of interest under the policy we would deploy.

\subsection{Why the heuristic fails}

Naively averaging judge scores is a heuristic that breaks in predictable ways:

\begin{itemize}
\item \textbf{Wrong scale.} Judge scores are not on your KPI scale; deltas can be misleading.
\item \textbf{Hidden drift and slice bias.} The meaning of a score can change across time, prompt families, or length; averages hide this.
\item \textbf{No uncertainty.} Without CIs it is hard to tell real uplifts from noise, especially at small label budgets.
\item \textbf{Off-policy illusion.} Reusing one judged log to assess many candidates without correction answers the wrong counterfactual (it reflects the logger, not the candidate).
\end{itemize}

\subsection{How CJE works}

\textbf{Causal Judge Evaluation (\cje)} turns judge scores into reliable estimates with confidence intervals. The core workflow:

\begin{enumerate}
\item \textbf{Calibrate:} Learn a mapping from judge scores to oracle labels on a small random sample (\autocal).
\item \textbf{Evaluate:} For on-policy comparison (\textbf{Direct Mode}), generate responses from each policy and average calibrated scores. For counterfactual inference (\textbf{Off-Policy}), reuse logged data with importance weighting.
\item \textbf{Report:} Confidence intervals that account for all uncertainty (\oua).
\end{enumerate}

Most users only need \textbf{Direct Mode}: generate outputs, calibrate judge scores, report CIs. For off-policy evaluation, \cje{} stabilizes importance weights (\simcal) and optionally adds a critic (\dr) for robustness. See §3 for details.

\subsection{Quick Start: Direct Mode for Model Comparison}

\textbf{Just want to compare models on an eval set?} This is the fastest path to reliable estimates with confidence intervals.

\paragraph{What you need:}
\begin{itemize}
\item 500--2000 prompts representative of deployment
\item Responses from each policy you want to compare
\item Judge scores for all responses (any scale works)
\item 50--150 oracle labels (ground truth on a random sample)
\end{itemize}

\paragraph{Steps:}
\begin{enumerate}
\item Generate outputs from each policy on the same prompts (paired sampling reduces variance).
\item Score all outputs with your judge (fixed version, frozen config).
\item Label a random sample of 50--150 examples with ground truth $Y$ (must be a simple random sample; stratify if subgroups matter).
\item Run \textbf{Direct Mode} (§2)---CJE learns the judge$\to$oracle calibration $f(S)$ automatically via \autocal.
\item Check five core diagnostics: coverage (§4.1), calibration (§4.2), judge stability (§4.3), OUA share (§4.4), transportability (§4.5).
\item Report estimates with 95\% CIs (includes \oua{} to account for calibrator uncertainty).
\end{enumerate}

\paragraph{What to read:}
\begin{itemize}
\item \textbf{Essential:} §2 (Direct Mode), §4.1--4.4 (core diagnostics), §6.1 (DM workflow).
\item \textbf{Skip for now:} §3 (off-policy methods), §4.5--4.7 (weight diagnostics).
\end{itemize}

You can return to off-policy methods (§3) later if you want counterfactual inference (``What if we deployed policy X?'') without regenerating outputs.

\subsection{Notation at a glance}

\begin{table}[h]
\centering
\caption{Key notation and terminology}
\label{tab:notation}
\begin{tabular}{ll}
\toprule
\textbf{Symbol/Term} & \textbf{Meaning} \\
\midrule
\multicolumn{2}{l}{\textit{Core symbols (all modes):}} \\
$X$ & Prompt (input) \\
$A$ & Response (action/output) \\
$S$ & Judge score $S = s(X, A) \in \mathbb{R}$ \\
$Y$ & Oracle label (ground truth outcome) $\in [0, 1]$ \\
$R$ & Calibrated reward $R = f(S) \in [0, 1]$ \\
\textbf{\autocal} & \textbf{AutoCal-R}: Automatic Calibration for Rewards \\
 & Isotonic map $f: S \to [0,1]$ learned from oracle slice \\
\textbf{\oua} & \textbf{OUA}: Oracle-Uncertainty Aware inference \\
 & Accounts for calibrator noise in confidence intervals \\
\textbf{\dm} & Direct Model (on-policy, fresh draws only) \\
\midrule
\multicolumn{2}{l}{\textit{Off-policy symbols (§3 only):}} \\
$\pi_0$ & Logging policy (baseline, deployed) \\
$\pi'$ & Candidate policy (target for evaluation) \\
$W_i$ & Raw importance weight $W_i = \pi'(A_i \mid X_i) / \pi_0(A_i \mid X_i)$ \\
$\tilde{w}_i$ & Stabilized, mean-one weight after \simcal \\
$w^{\text{sn}}_i$ & Self-normalized (Hájek) weight $w^{\text{sn}}_i = W_i / \bar{W}$ \\
\textbf{\simcal} & \textbf{SIMCal}: Score-Indexed Monotone Calibration \\
 & Stabilizes weights via monotone projection + variance cap \\
\textbf{\ips} & Importance Sampling (off-policy, logged data) \\
\textbf{\dr} & Doubly Robust (off-policy + critic/outcome model) \\
\bottomrule
\end{tabular}
\end{table}

\subsection{When you need more: Off-policy evaluation}

Direct Mode requires generating fresh outputs for each policy. If you want to answer counterfactual questions without regenerating---``What if we deployed policy X instead of our current baseline?''---you need off-policy methods:

\begin{itemize}
\item \textbf{Calibrated \ips{}} reuses a single logged dataset to evaluate multiple candidate policies by reweighting with importance ratios. Requires per-sequence log probabilities.

\item \textbf{Calibrated \dr{}} combines importance weighting with an outcome model for robustness and tighter confidence intervals. Requires both logged data and a small set of fresh draws.
\end{itemize}

Off-policy methods enable fast iteration (no regeneration), but introduce complexity (importance weights, overlap diagnostics). See §3 for full details, assumptions, and diagnostics. Most users should start with Direct Mode and adopt off-policy methods only when needed.

\subsection{Core diagnostics}

All CJE modes share five essential diagnostics that catch the most common failure modes:

\begin{enumerate}[label=(\alph*)]
\item \textbf{Score coverage:} Does the oracle slice span the evaluation $S$-range? (fix: add labels in uncovered bins)

\item \textbf{Calibration reliability:} Is $f(S)$ accurate across the $S$ spectrum? (fix: two-stage \autocal{} or add labels in problematic regions)

\item \textbf{Judge stability:} Is the judge's scoring consistent over time? (fix: freeze judge version, refresh oracle if drift detected)

\item \textbf{OUA share:} Is label scarcity the bottleneck for precision? (fix: add oracle labels in sparse regions)

\item \textbf{Transportability:} Does the calibrator transport across policies/time? (fix: refresh oracle or apply mean anchoring)
\end{enumerate}

These five checks apply whether you're using Direct Mode or off-policy methods. For off-policy evaluation (§3), three additional diagnostics are required:

\begin{enumerate}[label=(\alph*),resume]
\item \textbf{Effective sample size (ESS):} Do we have enough effective samples after reweighting? (fix: cohort restriction or switch to DR)

\item \textbf{Tail heaviness:} Are importance weights catastrophically heavy-tailed? (fix: SIMCal tuning or overlap-aware cohorting)

\item \textbf{DR orthogonality:} Is the outcome model orthogonal to the weights? (fix: improve critic or revisit weight calibration)
\end{enumerate}

See §4 for detailed thresholds, interpretation, and fixes for each diagnostic.

\subsection{What CJE provides}

\cje{} gives practitioners a complete workflow for reliable LLM evaluation:

\begin{itemize}
\item \textbf{Three analysis modes:} Direct Mode for on-policy comparison (simplest), Calibrated IPS for off-policy reuse, Calibrated DR for maximum accuracy when you have both logged data and fresh draws.

\item \textbf{Automatic calibration:} \autocal{} maps judge scores to oracle labels on a small random sample, putting estimates on the right scale.

\item \textbf{Honest confidence intervals:} \oua{} accounts for calibrator uncertainty, not just sampling noise.

\item \textbf{Eight high-leverage diagnostics:} Five core checks (coverage, calibration, judge stability, OUA share, transportability) plus three off-policy checks (ESS, tail heaviness, orthogonality). Each includes thresholds, interpretation, and concrete fixes.

\item \textbf{Off-policy robustness:} \simcal{} stabilizes importance weights via monotone projection, and DR adds an outcome model for double robustness.
\end{itemize}

\subsection{Scope and positioning}

This playbook focuses on practical implementation: assumptions in plain English, minimal recipes, diagnostics with actionable fixes, and reporting templates. Advanced material (theoretical guarantees, projection properties, semiparametric efficiency) lives in the appendix.

\cje{} builds on the treatment effect estimation literature, particularly work on efficient estimation with surrogates and limited outcome data \cite{kallus2024role}. Like that work, we avoid restrictive surrogacy conditions and instead study efficiency gains when surrogates supplement---rather than replace---primary outcomes under standard missing-at-random assumptions.

\subsection{Reader's guide}

\paragraph{For quick model comparison:} Start with the Quick Start (§1.5), then read §2 (Direct Mode), §4.1--4.4 (core diagnostics), and §6.1 (DM workflow). Skip §3 and §4.5--4.7 unless you later need counterfactual inference.

\paragraph{For comprehensive coverage:} We begin with \dm---the default when you can generate on shared prompts---then show how to reuse judged logs with Calibrated \ips{} and Calibrated \dr, followed by diagnostics, assumptions for causal interpretation, a practical playbook, compact case studies, and implementation notes.

\section{Direct Modeling (DM): Calibrated On-Policy Evaluation}

\subsection{What DM solves}

Direct Modeling answers: ``What KPI would we see on this prompt set if we shipped policy $\pi$?'' It is the closest offline analog to an A/B test: generate outputs for each policy on the same prompts (paired design), map judge scores to an outcome scale with \autocal, average the calibrated rewards, and attach confidence intervals that include calibration uncertainty (\oua).

\begin{quickref}
\textbf{DM in one minute (operator view)}
\begin{enumerate}
\item Fix the prompt set $\mathcal{X} = \{X_i\}_{i=1}^m$ for evaluation.
\item For each candidate policy $\pi$, generate one output $A_{\pi i}$ per prompt $X_i$ (use shared seeds where relevant).
\item Score each $(X_i, A_{\pi i})$ with the judge to get $S_{\pi i}$.
\item Calibrate scores to outcome-scale rewards via \autocal: $R_{\pi i} = f(S_{\pi i})$.
\item Estimate the value and (paired) differences with honest CIs (add \oua).
\end{enumerate}
\end{quickref}

\subsection{When to use DM}

\begin{itemize}
\item You can safely generate fresh outputs for every policy on a shared prompt set.
\item You want clean causal comparisons without dealing with propensities/overlap.
\item Your main risks are calibration coverage, judge drift, and label scarcity (handled by \autocal{} + diagnostics + \oua).
\end{itemize}

\subsection{Inputs \& setup}

\begin{description}
\item[Prompt set:] a fixed evaluation set; use the same $\mathcal{X}$ for all policies (paired design).
\item[Judge:] a scalar scoring function that returns $S = s(X, A)$ with fixed rubric/config.
\item[Oracle slice:] a small, diverse set with ground-truth labels $Y$ to train \autocal.
\end{description}

\subsection{AutoCal-R: map judge score to outcome}

\autocal{} learns a mean-preserving mapping $f: \mathbb{R} \to [0,1]$ from score $S$ to calibrated reward $R = f(S)$. By default it fits a monotone mapping; when needed, it auto-switches to a light two-stage index (e.g., spline index $\to$ isotonic) if regional miscalibration is detected.

\paragraph{Operator artifacts to check:}
\begin{itemize}
\item \textbf{Reliability:} out-of-fold calibration curve and regional error (low/mid/high $S$).
\item \textbf{Mean preservation:} oracle mean of $f(S)$ matches mean of $Y$.
\item \textbf{Score coverage:} fraction of evaluated $S$ that lies inside the oracle $S$-range; inspect boundary slopes.
\end{itemize}

\subsection{Estimator and paired contrasts}

With $R_{\pi i} = f(S_{\pi i})$ on shared prompts,
\begin{align}
\est{V}_{\dm}(\pi) &= \frac{1}{m} \sum_{i=1}^m R_{\pi i}, \\
\est{\Delta}(\pi, \pi') &= \frac{1}{m} \sum_{i=1}^m \left( R_{\pi i} - R_{\pi' i} \right).
\end{align}

Use paired standard errors for $\est{\Delta}$ (they are tighter because prompt-level noise cancels).

\subsection{Uncertainty that stays honest (OUA)}

Treating $f$ as fixed understates uncertainty when the oracle slice is small. \textbf{\oua{} (oracle-uncertainty aware)} inference refits $f$ across label folds and adds the induced variance:
\begin{equation}
\SE^2_{\text{total}} = \Var_{\text{main}} + \Var_{\oua},
\end{equation}
and reports 95\% CIs via the usual normal quantile. Always show the \oua{} share $\Var_{\oua}/\SE^2_{\text{total}}$.

\subsection{Diagnostics (highest leverage) \& quick fixes}

\begin{enumerate}[label=(\alph*)]
\item \textbf{Score coverage} (fraction of evaluated $S$ inside the labeled range; boundary slopes).

\emph{Fix:} add a small number of labels targeting uncovered $S$ bins; consider narrowing the prompt set to the intended deployment slice.

\item \textbf{Calibration reliability} (OOF curve + regional error).

\emph{Fix:} enable the two-stage \autocal{} fallback; add a coarse index (prompt family / length); collect a handful of labels in problematic regions.

\item \textbf{Judge stability / drift} (rank stability on a small anchor set; unchanged judge config).

\emph{Fix:} freeze judge version during the eval window; if drift appears, refresh the oracle slice and re-fit \autocal.

\item \textbf{\oua{} share} (how much labels drive the CI).

\emph{Fix:} if large, add labels (especially in sparse $S$ regions); otherwise more prompts help most.
\end{enumerate}

\subsection{Reporting template (DM)}

For each policy (and for key pairwise contrasts), report:

\begin{itemize}
\item Calibrated mean $\est{V}_{\dm}(\pi)$ with 95\% CI (paired for $\est{\Delta}$).
\item \oua{} share; note if paired design was used.
\item \autocal{} reliability snippet (with regional error) and the score-coverage badge.
\item Any drift checks performed on the judge.
\end{itemize}

\subsection{Sample size \& label planner (rules of thumb)}

Let $\hat{\sigma}_R^2$ be the prompt-level variance of $R$ for a policy on the evaluation set. Then a rough CI half-width for one policy is
\begin{equation}
\text{HalfWidth} \approx 1.96 \sqrt{\frac{\hat{\sigma}_R^2}{m} + \Var_{\oua}}.
\end{equation}

More prompts shrink the first term; more labels shrink the second (\oua). For pairwise contrasts, replace $\hat{\sigma}_R^2$ with the variance of $R_\pi - R_{\pi'}$ (often smaller due to pairing).

\subsection{Common pitfalls (and how to avoid them)}

\begin{itemize}
\item \textbf{Different prompts across policies.} Always evaluate on the same $\mathcal{X}$; match seeds if stochastic.

\item \textbf{Thin coverage at the edges of $S$.} Target labels to those bins or focus the prompt set; report the coverage badge.

\item \textbf{Calibration curve looks good overall but bad in one region.} Use two-stage \autocal{} or add a coarse index and a few labels.

\item \textbf{Large \oua{} share.} Label more---prioritize regions where $S$ is sparse or where policy comparisons matter.
\end{itemize}

\subsection{Minimal recipe (pseudocode)}

\begin{lstlisting}[language=Python,caption=DM Recipe]
# Inputs: prompts X, candidate policies Pi, judge s(.), oracle slice {(S, Y)}
# Output: calibrated means, paired contrasts, and CIs with OUA

# 1) Fit AutoCal-R on oracle (cross-fitted); persist reliability + coverage edges
f = fit_autocal_r(oracle_S, oracle_Y, K=5)

# 2) For each policy pi:
for pi in candidates:
    for i in range(m):
        A_pi[i] = generate(pi, X[i])     # same prompts for all
        S_pi[i] = judge_score(X[i], A_pi[i])
        R_pi[i] = f(S_pi[i])

    V_hat[pi] = np.mean(R_pi)

# 3) For contrasts (pi, pi'):
for (pi, pi_prime) in pairs:
    Delta_hat = np.mean(R_pi - R_pi_prime)    # paired difference
    SE_main = np.std(R_pi - R_pi_prime) / np.sqrt(m)

# 4) OUA: refit AutoCal-R across K folds of oracle; recompute step (2-3)
for k in range(K):
    f_k = fit_autocal_r(oracle_minus_fold_k)
    # ... recompute V_hat with f_k ...
SE_total_squared = SE_main**2 + Var_OUA

# 5) Report V_hat, Delta_hat, 95% CI using SE_total, OUA share,
#    reliability, coverage
\end{lstlisting}

\subsection{Scope notes}

\dm{} is on-policy for your chosen prompt distribution. If deployment prompts differ materially, treat that as a distribution-shift question: either adapt the prompt set to match deployment or build a small label slice representative of the target domain and re-fit \autocal.

\section{Off-Policy Evaluation: Calibrated IPS and DR}

\begin{mdframed}[linecolor=cjegray, backgroundcolor=white, linewidth=1pt]
\textbf{Note for Direct Mode users:} This section covers advanced off-policy methods for reusing logged data and answering counterfactual questions (``What if we deployed policy X?''). If you're just comparing policies on a fresh eval set using Direct Mode, you can \textbf{skip to §6 (Playbook)} or continue reading §4.1--4.4 (core diagnostics). Return here later if you need to reuse logged data without regenerating outputs.
\end{mdframed}

\subsection{What off-policy evaluation solves}

Off-policy evaluation answers: ``What KPI would we see if we deployed policy $\pi'$ instead of the logging policy $\pi_0$?'' It lets you assess multiple candidate policies by reusing a single judged log---no need to generate fresh outputs for each candidate. The core challenge is adjusting for the fact that the logged responses came from $\pi_0$, not from $\pi'$.

\begin{quickref}
\textbf{Off-policy in one minute (operator view)}
\begin{enumerate}
\item Collect a judged log under $\pi_0$: $(X_i, A_i, S_i)$ with logprobs for $\pi_0$ and candidate $\pi'$.
\item Calibrate scores to outcome-scale rewards: $R_i = f(S_i)$ via \autocal.
\item Compute importance weights: $W_i = \pi'(A_i \mid X_i) / \pi_0(A_i \mid X_i)$.
\item Stabilize weights with \simcal{} (monotone projection + variance cap).
\item Estimate value via \ips{} (weights only) or \dr{} (weights + critic).
\end{enumerate}
\end{quickref}

\subsection{When to use off-policy methods}

\begin{itemize}
\item You have a judged log from $\pi_0$ and want to assess multiple candidates $\pi'$ without regenerating.
\item You have per-sequence logprobs for both $\pi_0$ and candidate policies.
\item Your main risks are poor overlap (use diagnostics: ESS, tail index) and judge drift (check stability).
\item For \dr, you also need fresh draws to train a critic (outcome model).
\end{itemize}

\subsection{Calibrated IPS: reweight with stabilized weights}

The basic \ips{} estimator reweights calibrated rewards by the likelihood ratio:
\begin{equation}
\est{V}_{\text{IPS}}(\pi') = \frac{1}{n} \sum_{i=1}^n W_i \cdot R_i,
\end{equation}
where $W_i = \pi'(A_i \mid X_i) / \pi_0(A_i \mid X_i)$ and $R_i = f(S_i)$.

\paragraph{The variance problem.} When policies differ significantly, $W_i$ can have extreme values (heavy tails), leading to high variance and unstable estimates. A single large weight can dominate the sum.

\subsection{SIMCal: stabilize weights via monotone projection}

\textbf{\simcal{} (Surrogate Indexed Monotone Calibration)} stabilizes weights by projecting them onto monotone functions of the judge score $S$, then applying a variance cap:

\begin{enumerate}
\item \textbf{Fit three candidates} on a fold-out basis: baseline ($W$), monotone increasing ($W^\uparrow$), monotone decreasing ($W^\downarrow$).
\item \textbf{Blend} via cross-validated stacking to minimize variance.
\item \textbf{Variance cap:} if $\Var(W_{\text{blend}}) > \rho \cdot \Var(W)$, reproject to enforce $\Var \le \rho \cdot \Var(W)$.
\end{enumerate}

\textbf{Why it works:} Monotone projections weakly reduce variance by majorization, and the cap $\rho$ (default 0.95) ensures strict variance reduction. \simcal{} explicitly normalizes weights to mean one (so stabilized weights average to 1) and projects them as a monotone function of $S$ out-of-fold. Under our J2-M/J2-MX assumptions (score-indexed monotone sufficiency), this projection \emph{stabilizes variance with approximately unbiased mean-one normalization}. We surface cohort and trimming sensitivities in diagnostics to reveal any residual bias; if J2-M fails or overlap is poor, these sensitivity checks will flag the issue.

\paragraph{Operator artifacts to check:}
\begin{itemize}
\item \textbf{ESS (Effective Sample Size):} $({\textstyle\sum} \tilde{w})^2 / ({\textstyle\sum} \tilde{w}^2)$, where $\tilde{w}$ are post-\simcal{} weights. Aim for ESS $\ge 30\%$ of $n$ (PASS); 10--30\% is acceptable (WARN); $< 10\%$ is poor (FAIL); $< 1\%$ is catastrophic (REFUSE).
\item \textbf{Tail heaviness:} Hill index $\alpha \ge 2$ (finite variance). If $\alpha < 2$, weights are catastrophic.
\item \textbf{Weight summary:} min, median, max, 95th percentile before/after \simcal.
\end{itemize}

\subsection{Calibrated DR: add an outcome model for robustness}

Doubly robust (DR) combines \ips{} with an outcome model to gain efficiency and robustness. CJE supports two DR implementations, both valid under different modeling assumptions.

\paragraph{Two DR implementations.}

\emph{(A) Score-only AIPW (lightweight, default).} Fit $\hat{g}(S) \approx \E[R \mid S]$ on fresh draws from $\pi'$ using the judge score $S$ only. Then:
\begin{equation}
\est{V}_{\text{DR-S}}(\pi') = \frac{1}{n} \sum_{i=1}^n \left[ \tilde{w}_i \cdot (R_i - \hat{g}(S_i)) + \hat{g}(S_i) \right],
\end{equation}
where $\tilde{w}_i$ are the stabilized, mean-one weights after \simcal, and $\hat{g}$ is typically a monotone function of $S$ (isotonic regression). This is the default in CJE: low cost, works well when $S$ is a strong predictor.

\emph{(B) Action-conditioned DR (canonical, research).} Fit $\hat{q}(X,A) \approx \E[R \mid X, A]$ using prompt features $X$ and response $A$, then compute $\hat{g}_{\pi'}(X) = \E_{A \sim \pi'(\ \cdot \mid X)} \hat{q}(X, A)$ via Monte Carlo rollouts or analytic evaluation. Then:
\begin{equation}
\est{V}_{\text{DR}}(\pi') = \frac{1}{n} \sum_{i=1}^n \left[ \hat{g}_{\pi'}(X_i) + \tilde{w}_i \cdot (R_i - \hat{q}(X_i, A_{0i})) \right].
\end{equation}
This is the canonical DR estimator from the semiparametric literature; use it when you have rich features or need maximum robustness (e.g., when $S$ alone is insufficient).

\textbf{When to use which:}
\begin{itemize}
\item \textbf{DR-S (score-only):} When judge scores are informative and you want low implementation cost. Default for most LLM evaluations.
\item \textbf{DR (action-conditioned):} When you have structured prompt features (length, domain, difficulty), or when $S$ is a weak predictor and you need the full robustness guarantee.
\end{itemize}

\paragraph{Double robustness guarantee:} Both estimators are consistent if \emph{either} the stabilized weights $\tilde{w}(\pi')$ converge to the true density ratios (up to mean-one normalization) \emph{or} the outcome model ($\hat{g}$ or $\hat{q}$) is correctly specified; if both are good, DR attains the semiparametric efficiency bound.

\paragraph{Why isotonic regression for $\hat{g}(S)$?} The default outcome model uses isotonic regression because it enforces exactly the right structural prior: \emph{higher judge score $\Rightarrow$ no worse expected reward}. This minimal monotonicity assumption avoids misspecification risk from rigid parametric forms (sigmoid, beta), provides mean preservation by construction (critical for unbiased DR), and achieves strong stability with few fresh draws (5-10\% coverage often sufficient). Isotonic's step-function output also makes edge fragility visible in diagnostics, enabling targeted label collection. When monotonicity fails (e.g., systematic length bias at fixed $S$), \autocal{}'s two-stage variant learns a smooth transformation first.

\subsection{Inputs \& setup (off-policy)}

\begin{description}
\item[Judged log:] Responses from $\pi_0$ with judge scores $S$ and logprobs for $\pi_0$ and each candidate $\pi'$.
\item[Oracle slice:] A small random subsample with ground truth $Y$ for \autocal{} (same as DM).
\item[Fresh draws (DR only):] For each candidate $\pi'$, generate fresh responses on the same prompts to train the critic $\hat{g}(X)$.
\item[Judge:] Same fixed rubric/config; check stability over time (Kendall $\tau$ on an anchor set).
\end{description}

\subsection{Fresh draws for outcome modeling}

For \dr, the outcome model $\hat{g}(X)$ must predict what $\pi'$ would achieve. Train it on \textbf{fresh draws}: responses generated by $\pi'$ on the same prompts, with judge scores.

Without fresh draws, you cannot build a valid outcome model, and \ips{} is your only option.

\subsection{Estimator and inference}

With calibrated rewards $R_i = f(S_i)$ and stabilized, mean-one weights $\tilde{w}_i$ (via \simcal), compute:
\begin{align}
\est{V}_{\text{IPS}}(\pi') &= \frac{1}{n} \sum_{i=1}^n \tilde{w}_i \cdot R_i, \\
\est{V}_{\text{DR-S}}(\pi') &= \frac{1}{n} \sum_{i=1}^n \left[ \tilde{w}_i \cdot (R_i - \hat{g}(S_i)) + \hat{g}(S_i) \right].
\end{align}

Standard errors are computed via influence functions (per-sample contributions), which account for calibration, cross-fitting, and oracle uncertainty (via \oua). Calibrated IPS uses \emph{outer cross-validation} for \simcal{} selection by default, so weight-learning uncertainty is reflected in the SEs. CIs are formed using the normal approximation for large samples or t-critical with Satterthwaite df when clusters/oracle folds are small (see \S\ref{sec:uncertainty}).

\subsection{Diagnostics (highest leverage) \& quick fixes}

\begin{enumerate}[label=(\alph*)]
\item \textbf{ESS (Effective Sample Size)} after \simcal.

\emph{Fix:} If ESS $< 30\%$ (WARN), try cohort restriction (focus on prompts with better overlap) or tune \simcal{} variance cap $\rho$. If ESS $< 10\%$ (FAIL), switch to \dr{} with a stronger outcome model. If ESS $< 1\%$ (REFUSE), the estimate is unreliable---regenerate or narrow scope.

\item \textbf{Tail heaviness} (Hill index $\alpha$).

\emph{Fix:} If $\alpha < 2$, weights have infinite variance. Use \simcal{} aggressively, restrict to high-overlap cohorts, or switch to \dr. Check for provider temperature/frequency-penalty drift.

\item \textbf{Weight stability} (compare raw vs.\ \simcal-adjusted distributions).

\emph{Fix:} Large changes indicate strong reliance on monotonicity assumption (J2-M). Validate by checking that $\E[Y \mid S]$ and $\E[W \mid S]$ are indeed monotone on the oracle slice.

\item \textbf{DR orthogonality score} (empirical moment with CI).

\emph{Fix:} If the orthogonality CI does not contain zero, either the outcome model or the weights are poor. Improve outcome model regularization, add fresh draws, or revisit \simcal{} / overlap diagnostics.

\item \textbf{Oracle uncertainty share} (fraction of total variance from \oua).

\emph{Fix:} If large, add labels targeting uncovered $S$ regions or high-variance prompt families.
\end{enumerate}

\subsection{Reporting template (off-policy)}

For each candidate policy, report:

\begin{itemize}
\item Calibrated mean $\est{V}(\pi')$ with 95\% CI (\ips{} or \dr).
\item ESS (absolute and as \% of $n$), Hill index $\alpha$.
\item Weight summary: min, median, max, 95th percentile (raw and post-\simcal).
\item For \dr, orthogonality score with CI.
\item \oua{} share; note if oracle slice coverage is thin.
\item Any judge stability checks (e.g., Kendall $\tau$ on anchor set).
\end{itemize}

\subsection{Sample size \& label planner (off-policy)}

Let $\hat{\sigma}_W^2$ be the variance of $W_i \cdot R_i$ for \ips, or the variance of the DR influence function. Then a rough CI half-width is
\begin{equation}
\text{HalfWidth} \approx 1.96 \sqrt{\frac{\hat{\sigma}_W^2}{n} + \Var_{\oua}}.
\end{equation}

For \ips, more data helps only if ESS is not the bottleneck; if ESS is low, improving overlap (via cohort restriction or stronger \simcal) is more effective than adding samples. For \dr, more fresh draws improve the outcome model and tighten intervals.

\subsection{When to use IPS vs.\ DR}

\begin{itemize}
\item \textbf{Use \ips{}} when you cannot generate fresh draws, or when overlap is excellent (ESS $\ge 30\%$, PASS) and variance is low.
\item \textbf{Use \dr{}} when you can afford fresh draws and overlap is moderate to poor (ESS $< 30\%$, WARN/FAIL). \dr{} is more robust and typically yields tighter CIs.
\item \textbf{Use stacked-DR} (ensemble of DR variants) when you want the best of all worlds: robustness, efficiency, and automatic selection among multiple outcome models.
\end{itemize}

\subsection{Common pitfalls (and how to avoid them)}

\begin{itemize}
\item \textbf{Ignoring ESS.} A single large weight can dominate. Always check ESS; if $< 30\%$ (WARN), investigate via diagnostics; if $< 10\%$ (FAIL), restrict scope or switch to \dr.

\item \textbf{Assuming weights have finite variance.} Check the Hill index $\alpha$; if $< 2$, estimates are unstable. Use \simcal{} or cohort restriction.

\item \textbf{Reusing logged responses for the outcome model.} The outcome model must be trained on \emph{fresh draws} from $\pi'$, not on $\pi_0$'s responses. Otherwise \dr{} fails.

\item \textbf{Ignoring judge drift.} If the judge's scoring changes over time, scores from the log are not comparable to fresh scores. Check stability via an anchor set.

\item \textbf{Thin oracle coverage.} If labeled $S$ does not cover the range of evaluated $S$, \autocal{} extrapolates poorly. Target labels to uncovered bins or report the coverage badge.
\end{itemize}

\subsection{Minimal recipe (pseudocode: IPS)}

\begin{lstlisting}[language=Python,caption=Calibrated IPS Recipe]
# Inputs: judged log from pi_0 with (X, A, S, logprob_pi0, logprob_pi_prime)
#         oracle slice {(S, Y)}
# Output: calibrated mean, ESS, CI with OUA

# 1) Fit AutoCal-R on oracle (cross-fitted)
f = fit_autocal_r(oracle_S, oracle_Y, K=5)

# 2) Calibrate rewards and compute raw weights
for i in range(n):
    R[i] = f(S[i])
    W[i] = exp(logprob_pi_prime[i] - logprob_pi0[i])

# 3) Stabilize weights with SIMCal (OOF stacking + variance cap)
W_calibrated = simcal(W, S, rho=0.95, K=5)

# 4) Estimate value
V_hat_ips = np.mean(W_calibrated * R)

# 5) Compute ESS and diagnostics
ESS = (np.sum(W_calibrated)**2) / np.sum(W_calibrated**2)
alpha_hill = estimate_hill_index(W_calibrated)

# 6) OUA: refit AutoCal-R across K folds and add oracle variance
for k in range(K):
    f_k = fit_autocal_r(oracle_minus_fold_k)
    # ... recompute V_hat with f_k ...
SE_total_squared = SE_main**2 + Var_OUA

# 7) Report V_hat, 95% CI, ESS, Hill index, OUA share
\end{lstlisting}

\subsection{Minimal recipe (pseudocode: DR)}

\begin{lstlisting}[language=Python,caption=Calibrated DR Recipe]
# Inputs: judged log from pi_0, oracle slice, fresh draws from pi_prime
# Output: calibrated mean, orthogonality score, CI with OUA

# 1) Fit AutoCal-R on oracle
f = fit_autocal_r(oracle_S, oracle_Y, K=5)

# 2) Calibrate rewards and compute stabilized weights (as in IPS)
for i in range(n):
    R[i] = f(S[i])
    W[i] = exp(logprob_pi_prime[i] - logprob_pi0[i])
W_calibrated = simcal(W, S, rho=0.95, K=5)

# 3) Train outcome model g(S) on fresh draws (cross-fitted)
for k in range(K):
    train_scores = fresh_draws_scores_minus_fold_k
    g_k = fit_outcome_model(train_scores, R_fresh)  # isotonic regression
    # Predict on fold k:
    for i in fold_k:
        g_hat[i] = g_k(S[i])

# 4) Compute DR estimate
V_hat_dr = np.mean(W_calibrated * (R - g_hat) + g_hat)

# 5) Orthogonality check (empirical moment: E[W * (R - g_hat)])
ortho_moment = np.mean(W_calibrated * (R - g_hat))
ortho_se = np.std(W_calibrated * (R - g_hat)) / np.sqrt(n)
ortho_ci = [ortho_moment - 1.96*ortho_se, ortho_moment + 1.96*ortho_se]

# 6) OUA and CI (as in IPS)
# ... refit AutoCal-R across folds ...
SE_total_squared = SE_main**2 + Var_OUA

# 7) Report V_hat_dr, orthogonality CI, ESS, OUA share
\end{lstlisting}

\subsection{Scope notes}

Off-policy evaluation assumes that the logging policy $\pi_0$ had positive probability of generating any response that $\pi'$ might generate (overlap, assumption D2). If policies are too different (e.g., different model families with non-overlapping support), off-policy methods fail. In such cases, generate fresh outputs and use \dm{} instead.

Judge stability is also critical: if the judge's meaning of a score changes between the time the log was collected and the time fresh draws are evaluated, calibration breaks. Always check drift via an anchor set with stable prompts.

\section{Diagnostics: The Five High-Leverage Checks}

\subsection{Overview}

CJE's diagnostic dashboard is designed to catch the most important failure modes with minimal operator overhead. Each diagnostic points directly to a concrete fix. This section details the five checks, their interpretation, thresholds, and remediations.

\subsection{The diagnostic checklist}

\begin{enumerate}
\item \textbf{Score coverage} (for all modes): Does the oracle slice span the evaluation $S$-range?
\item \textbf{Calibration reliability} (for all modes): Is $f(S)$ accurate across the $S$ spectrum?
\item \textbf{Effective sample size (ESS)} (for \ips/\dr): Do we have enough effective samples after reweighting?
\item \textbf{Tail heaviness} (for \ips/\dr): Are importance weights catastrophically heavy-tailed?
\item \textbf{DR orthogonality} (for \dr): Is the critic good enough for doubly robust guarantees?
\end{enumerate}

Each diagnostic produces a compact artifact (a number, a CI, or a plot) and a traffic-light signal (pass / warn / fail).

\subsection{Diagnostic 1: Score coverage}

\paragraph{What it measures.} The fraction of evaluated scores $S$ that fall within the range of oracle scores used to train \autocal. Also checks boundary slopes for extrapolation risk.

\paragraph{Why it matters.} If \autocal{} must extrapolate beyond the labeled $S$-range, calibrated rewards $R = f(S)$ can be wildly wrong. This is the single most common source of large errors.

\paragraph{Artifacts to inspect:}
\begin{itemize}
\item \textbf{Coverage fraction:} $\#\{i : S_{\min}^{\text{oracle}} \le S_i \le S_{\max}^{\text{oracle}}\} / n$.
\item \textbf{Boundary slopes:} slope of $f$ near $S_{\min}$ and $S_{\max}$.
\item \textbf{Histogram overlay:} oracle $S$ distribution vs.\ evaluation $S$ distribution.
\end{itemize}

\paragraph{Thresholds:}
\begin{itemize}
\item \textbf{Pass:} Coverage $\ge 95\%$, boundary slopes moderate.
\item \textbf{Warn:} Coverage $\in [85\%, 95\%)$ or steep boundary slopes.
\item \textbf{Fail:} Coverage $< 85\%$ or nearly-flat/nearly-vertical boundary slopes.
\end{itemize}

\paragraph{Fixes:}
\begin{itemize}
\item Add a small number of labels (5--20) targeting the uncovered $S$ bins.
\item Narrow the prompt set to the intended deployment slice.
\item If the evaluation set is meant to stress-test edge cases, ensure the oracle slice includes examples from those edges.
\end{itemize}

\subsection{Diagnostic 2: Calibration reliability}

\paragraph{What it measures.} Out-of-fold (OOF) calibration accuracy: how well $f(S)$ predicts $Y$ on held-out oracle data. Includes overall error and regional error (low/mid/high $S$).

\paragraph{Why it matters.} Even if coverage is good, \autocal{} can be miscalibrated in specific regions (e.g., $f(S)$ is too optimistic for high $S$). Regional miscalibration biases estimates and breaks mean preservation.

\paragraph{Artifacts to inspect:}
\begin{itemize}
\item \textbf{OOF calibration curve:} scatter plot of $f(S)$ vs.\ $Y$ on held-out folds, with diagonal reference.
\item \textbf{Mean preservation check:} $|\E_{\text{oracle}}[f(S)] - \E_{\text{oracle}}[Y]|$ (should be $\approx 0$).
\item \textbf{Regional error:} mean absolute error (MAE) in low, mid, high $S$ terciles.
\end{itemize}

\paragraph{Thresholds:}
\begin{itemize}
\item \textbf{Pass:} OOF MAE $< 0.05$, regional MAE balanced, mean preservation within $\pm 0.02$.
\item \textbf{Warn:} OOF MAE $\in [0.05, 0.10]$ or one region has $2\times$ the error of others.
\item \textbf{Fail:} OOF MAE $> 0.10$ or severe regional imbalance or mean drift $> 0.05$.
\end{itemize}

\paragraph{Fixes:}
\begin{itemize}
\item Enable two-stage \autocal{} (spline index $\to$ isotonic) for regional adaptivity.
\item Add a coarse index (prompt family, length bin) to allow family-specific calibration.
\item Collect 10--30 additional labels in the problematic $S$ region.
\item Check if judge drift occurred (see judge stability diagnostic).
\end{itemize}

\subsection{Diagnostic 3: Effective sample size (ESS)}

\paragraph{What it measures.} The number of ``effective'' independent samples after importance weighting:
\begin{equation}
\text{ESS} = \frac{\big(\sum_{i=1}^n \tilde{w}_i\big)^2}{\sum_{i=1}^n \tilde{w}_i^2},
\end{equation}
where $\tilde{w}_i$ are the stabilized, mean-one weights after \simcal. ESS ranges from 1 (one weight dominates) to $n$ (all weights equal).

\paragraph{Why it matters.} Low ESS means a few samples dominate the estimate, leading to high variance and unreliable CIs. ESS is the primary diagnostic for off-policy viability.

\paragraph{Artifacts to inspect:}
\begin{itemize}
\item \textbf{ESS (absolute and \%)}: report both raw ESS and ESS$/n$.
\item \textbf{Weight distribution:} histogram or quantiles (min, median, 95th percentile, max) before and after \simcal.
\item \textbf{Weight vs.\ $S$ scatter:} check if weights are monotone in $S$ (validates J2-M assumption).
\end{itemize}

\paragraph{Thresholds:}
\begin{itemize}
\item \textbf{Pass:} ESS $\ge 30\%$ of $n$ (excellent overlap).
\item \textbf{Warn:} ESS $\in [10\%, 30\%)$ (moderate overlap; \dr{} recommended).
\item \textbf{Fail:} ESS $< 10\%$ (poor overlap; estimates unreliable unless \dr{} with strong outcome model).
\item \textbf{Refuse:} ESS $< 1\%$ (catastrophic; do not trust any estimate).
\end{itemize}

\paragraph{Fixes:}
\begin{itemize}
\item Use \simcal{} with aggressive variance cap $\rho$ (e.g., 0.90).
\item Restrict to a high-overlap cohort (e.g., prompts where $|\log \tilde{w}_i| < 2$).
\item Switch from \ips{} to \dr{} with a strong outcome model.
\item If ESS $< 1\%$, regenerate fresh outputs for the candidate policy and use \dm{} instead.
\end{itemize}

\subsection{Diagnostic 4: Tail heaviness (Hill index)}

\paragraph{What it measures.} The tail index $\alpha$ of the weight distribution, estimated via the Hill estimator. If $\alpha < 2$, the weights have infinite variance; if $\alpha < 1$, infinite mean.

\paragraph{Why it matters.} Heavy-tailed weights break standard inference. Even if ESS looks reasonable, $\alpha < 2$ means variance estimates are unreliable and CIs can be arbitrarily wrong.

\paragraph{Artifacts to inspect:}
\begin{itemize}
\item \textbf{Hill index $\alpha$:} estimated from the upper tail of $\tilde{w}$ (post-\simcal{} weights).
\item \textbf{QQ-plot:} compare weight quantiles to Pareto($\alpha$) reference.
\item \textbf{Max/median ratio:} if $> 100$, likely heavy-tailed.
\end{itemize}

\paragraph{Thresholds:}
\begin{itemize}
\item \textbf{Pass:} $\alpha \ge 3$ (well-behaved tails).
\item \textbf{Warn:} $\alpha \in [2, 3)$ (finite variance, but higher moments unstable).
\item \textbf{Fail:} $\alpha < 2$ (infinite variance; estimates unstable).
\item \textbf{Critical:} $\alpha < 1$ (infinite mean; nonsensical).
\end{itemize}

\paragraph{Fixes:}
\begin{itemize}
\item Apply \simcal{} with a strict variance cap $\rho < 0.95$.
\item Cohort restriction to high-overlap prompts.
\item Check for provider drift (temperature, frequency penalty) that might cause unexpected tail behavior.
\item If $\alpha < 2$ persists, switch to \dr{} or regenerate.
\end{itemize}

\subsection{Diagnostic 5: DR orthogonality score}

\paragraph{What it measures.} The empirical moment
\begin{equation}
\text{OrthoScore} = \frac{1}{n} \sum_{i=1}^n \tilde{w}_i \cdot (R_i - \hat{g}(S_i)),
\end{equation}
with a 95\% CI, where $\tilde{w}_i$ are the stabilized, mean-one weights after \simcal. Under ideal conditions (mean-one weights and well-specified outcome model), this should be near zero.

\paragraph{Why it matters.} Orthogonality is the key to doubly robust inference. If the orthogonality score is far from zero, either the weights or the outcome model (or both) are poor, and \dr{} loses its efficiency and robustness guarantees.

\paragraph{Artifacts to inspect:}
\begin{itemize}
\item \textbf{Orthogonality score with CI:} point estimate $\pm 1.96 \cdot \SE$.
\item \textbf{Residual plot:} $(R_i - \hat{g}(S_i))$ vs.\ $\tilde{w}_i$ to spot patterns.
\item \textbf{Outcome model performance:} OOF $R^2$ or MAE on fresh draws.
\end{itemize}

\paragraph{Thresholds:}
\begin{itemize}
\item \textbf{Pass:} 95\% CI contains zero, $|$OrthoScore$| < 0.05$.
\item \textbf{Warn:} CI contains zero, but $|$OrthoScore$| \in [0.05, 0.10]$.
\item \textbf{Fail:} CI does not contain zero or $|$OrthoScore$| > 0.10$.
\end{itemize}

\paragraph{Fixes:}
\begin{itemize}
\item Improve the outcome model: add regularization, use a more flexible model (e.g., two-stage isotonic), or increase fresh draw sample size.
\item Revisit \simcal{} settings or overlap diagnostics (poor weights can break orthogonality).
\item Check cross-fitting: ensure folds are balanced and representative.
\item If orthogonality fails persistently, fall back to \ips{} or regenerate for \dm.
\end{itemize}

\subsection{Judge stability diagnostic (supplementary)}

\paragraph{What it measures.} Rank stability of judge scores on a small anchor set (20--50 fixed prompts) over time, measured via Kendall $\tau$ or Spearman $\rho$.

\paragraph{Why it matters.} If the judge's interpretation of scores drifts between the time the log was collected and the time fresh draws are evaluated, calibration breaks. This is especially critical for off-policy evaluation.

\paragraph{Artifacts to inspect:}
\begin{itemize}
\item \textbf{Rank correlation:} Kendall $\tau$ between scores at time $t_0$ (logging) and $t_1$ (evaluation).
\item \textbf{Score drift plot:} scatter of $S_{t_0}$ vs.\ $S_{t_1}$ for anchor prompts.
\end{itemize}

\paragraph{Thresholds:}
\begin{itemize}
\item \textbf{Pass:} $\tau \ge 0.90$ (strong stability).
\item \textbf{Warn:} $\tau \in [0.75, 0.90)$ (moderate drift; check config).
\item \textbf{Fail:} $\tau < 0.75$ (severe drift; refresh oracle and re-calibrate).
\end{itemize}

\paragraph{Fixes:}
\begin{itemize}
\item Freeze judge version and config during the evaluation window.
\item If drift is detected, refresh the oracle slice with new labels and re-fit \autocal.
\item Consider using a judging protocol with locked temperature/sampling to reduce stochasticity.
\end{itemize}

\paragraph{Automated temporal drift detection.} When evaluation data includes timestamps, CJE can automatically detect drift over time using \texttt{timestamp\_field} and \texttt{check\_drift=True}. This sorts samples by timestamp, computes sequential Kendall $\tau$ correlations across time batches, and flags drift points. This automates the anchor-set workflow above for datasets with temporal metadata.

\subsection{Oracle uncertainty (OUA) share diagnostic}

\paragraph{What it measures.} The fraction of total variance attributable to oracle uncertainty (calibrator noise):
\begin{equation}
\text{OUA share} = \frac{\Var_{\oua}}{\SE^2_{\text{total}}}.
\end{equation}

\paragraph{Why it matters.} High OUA share means label scarcity is the bottleneck, not prompt scarcity. Adding more prompts won't help; you need more labels.

\paragraph{Artifacts to inspect:}
\begin{itemize}
\item \textbf{OUA share:} reported as a percentage.
\item \textbf{Variance decomposition:} $\SE^2_{\text{total}} = \Var_{\text{main}} + \Var_{\oua}$.
\end{itemize}

\paragraph{Thresholds:}
\begin{itemize}
\item \textbf{Pass:} OUA share $< 20\%$ (labels sufficient).
\item \textbf{Warn:} OUA share $\in [20\%, 50\%]$ (labels becoming a bottleneck).
\item \textbf{Fail:} OUA share $> 50\%$ (labels are the main source of uncertainty).
\end{itemize}

\paragraph{Fixes:}
\begin{itemize}
\item Add labels, prioritizing regions where $S$ is sparse or where policy comparisons are sensitive.
\item For high OUA share, adding prompts yields diminishing returns; focus labeling budget on coverage and regional balance.
\end{itemize}

\subsection{Traffic-light summary and refusal gates}

CJE estimators can optionally enforce \textbf{refusal gates}: if a diagnostic crosses a critical threshold, the estimator returns \texttt{NaN} instead of a potentially misleading number.

\paragraph{Unified diagnostic thresholds.}

Table~\ref{tab:thresholds} summarizes all diagnostic thresholds in one place. Use this as the single source of truth for interpreting diagnostic outputs.

\begin{table}[h]
\centering
\caption{Unified diagnostic thresholds for CJE}
\label{tab:thresholds}
\begin{tabular}{lccc|c}
\toprule
\textbf{Diagnostic} & \textbf{PASS} & \textbf{WARN} & \textbf{FAIL} & \textbf{REFUSE} \\
\midrule
ESS / $n$ (post-\simcal) & $\ge 30\%$ & $[10\%, 30\%)$ & $< 10\%$ & $< 1\%$ \\
Hill index $\alpha$ & $\ge 3$ & $[2, 3)$ & $< 2$ & $< 1$ \\
Score coverage & $\ge 95\%$ & $[85\%, 95\%)$ & $< 85\%$ & — \\
Calibration OOF MAE & $< 0.05$ & $[0.05, 0.10]$ & $> 0.10$ & — \\
Mean preservation & $< 0.02$ & $[0.02, 0.05]$ & $> 0.05$ & — \\
DR orthogonality $|$score$|$ & $< 0.05$ & $[0.05, 0.10]$ & $> 0.10$ & — \\
\quad (CI must contain 0) & \checkmark & \checkmark & $\times$ & — \\
OUA share & $< 20\%$ & $[20\%, 50\%]$ & $> 50\%$ & — \\
Judge stability $\tau$ & $\ge 0.90$ & $[0.75, 0.90)$ & $< 0.75$ & — \\
\bottomrule
\end{tabular}
\end{table}

\paragraph{Default refusal gates:}
\begin{itemize}
\item ESS $< 1\%$ $\to$ refuse to estimate (off-policy only).
\item Hill index $\alpha < 2$ $\to$ refuse (off-policy only).
\item Coverage $< 50\%$ $\to$ warn strongly (all modes).
\end{itemize}

Operators can adjust these gates or disable them for exploratory analysis, but the default is to fail loudly rather than produce garbage.

\subsection{Diagnostic workflow (flowchart)}

\begin{enumerate}
\item \textbf{Check score coverage.} If $< 85\%$, add labels or narrow scope. If pass, continue.
\item \textbf{Check calibration reliability.} If regional error is high, enable two-stage \autocal{} or add labels. If pass, continue.
\item \textbf{(Off-policy only) Check ESS.} If $< 30\%$ (WARN), review diagnostics and consider \dr. If $< 10\%$ (FAIL), use \simcal{} or switch to \dr. If $< 1\%$ (REFUSE), regenerate. If pass, continue.
\item \textbf{(Off-policy only) Check tail index.} If $\alpha < 2$, apply stricter \simcal{} or cohort restriction. If pass, continue.
\item \textbf{(DR only) Check orthogonality.} If CI excludes zero, improve outcome model or weights. If pass, trust the estimate.
\item \textbf{Check OUA share.} If $> 50\%$, add labels. If pass, you're done.
\end{enumerate}

If any diagnostic fails and the fix is not immediately available, report the failure and do not ship the estimate.

\subsection{Visualization gallery (artifacts)}

For full audit trails, CJE produces:
\begin{itemize}
\item \textbf{Calibration curve:} $f(S)$ vs.\ $Y$ with OOF fit and mean-preservation line.
\item \textbf{Coverage histogram:} oracle $S$ vs.\ evaluation $S$ distributions.
\item \textbf{Weight distribution:} before/after \simcal, with ESS and Hill index annotations.
\item \textbf{Orthogonality residual plot:} $(R - \hat{g}(S))$ vs.\ $\tilde{w}$ with CI band.
\item \textbf{OUA variance decomposition:} stacked bar showing $\Var_{\text{main}}$ vs.\ $\Var_{\oua}$.
\end{itemize}

These plots are auto-generated by the CJE visualization module and can be included in reports or dashboards.

\subsection{Example: interpreting a diagnostic report}

\begin{lstlisting}
=== CJE Diagnostic Report ===
Mode: Calibrated DR
Estimator: stacked-dr
Target policy: gpt-4-mini
Baseline policy: gpt-3.5-turbo

[1] Score Coverage: 92% (PASS)
    Oracle S range: [2.1, 9.8]
    Eval S range: [1.9, 9.9]
    Boundary slopes: moderate

[2] Calibration Reliability: OOF MAE = 0.04 (PASS)
    Mean preservation: -0.01 (excellent)
    Regional MAE: low=0.03, mid=0.04, high=0.05 (balanced)

[3] ESS: 342 / 1000 = 34% (WARN)
    Raw ESS: 18%
    Post-SIMCal ESS: 34%
    Recommendation: acceptable for DR; consider cohort restriction if CI too wide

[4] Tail Index: alpha = 2.8 (PASS)
    Max/median weight ratio: 12.3
    Tail behavior: well-behaved

[5] DR Orthogonality: 0.03 [-0.02, 0.08] (PASS)
    CI contains zero
    Outcome model OOF R^2: 0.67

[6] OUA Share: 15% (PASS)
    Labels sufficient; variance dominated by prompt variation

Overall: PASS (with ESS warning)
Recommendation: Estimate is reliable. Consider adding fresh draws or cohort restriction for tighter CIs.
\end{lstlisting}

\subsection{Summary}

The five diagnostics (coverage, reliability, ESS, tail index, orthogonality) catch the most common failure modes. Each produces a compact artifact, a traffic-light signal, and a concrete fix. Always inspect diagnostics before trusting estimates, and always report them alongside results.

\section{Assumptions for Causal Interpretation}

\subsection{Overview}

CJE estimates answer causal questions: ``What would happen if we deployed $\pi'$?'' To interpret estimates causally, we need assumptions. This section states them in plain English, explains when they matter, how to check them, and what happens when they fail.

\subsection{Data assumptions}

\begin{assumption}[SUTVA: Stable Unit Treatment Value]
\label{assum:sutva}
The outcome for prompt $X_i$ under policy $\pi$ does not depend on what policy was applied to other prompts. Formally: no interference and no hidden versions of treatment.
\end{assumption}

\paragraph{Plain English.} Evaluating one prompt doesn't affect another. This rules out settings where responses are shown to users sequentially and influence each other (e.g., conversational context bleeding across prompts).

\paragraph{When it matters.} All modes (\dm, \ips, \dr). Violations are rare in offline batch evaluation but common in online settings with statefulness.

\paragraph{How to check.} SUTVA is mostly a design question: ensure prompts are evaluated independently. In practice, use fixed seeds, separate sessions, or randomize order.

\paragraph{What happens if it fails.} Estimates are biased in unknown directions. The only fix is to redefine the estimand to account for interference (beyond this playbook's scope).

\begin{assumption}[Overlap (Positivity)]
\label{assum:overlap}
For all prompts $X$ and responses $A$ such that $\pi'(A \mid X) > 0$, we have $\pi_0(A \mid X) > 0$. Informally: the logging policy could have generated any response the target policy might generate.
\end{assumption}

\paragraph{Plain English.} Off-policy evaluation requires that the logged data ``covers'' the target policy's behavior. If $\pi'$ generates responses that $\pi_0$ would never produce, we have no data to learn from.

\paragraph{When it matters.} Off-policy only (\ips, \dr). Not needed for \dm{} (you generate fresh outputs).

\paragraph{How to check.} ESS and tail diagnostics. Low ESS ($< 10\%$) or heavy tails ($\alpha < 2$) indicate poor overlap. Scatter plot of $\log \pi'(A \mid X) / \pi_0(A \mid X)$ vs.\ $X$ can reveal systematic gaps.

\paragraph{What happens if it fails.} Importance weights explode, variance becomes infinite, estimates are dominated by a few outliers. The only fix is to restrict to a high-overlap cohort, switch to \dr{} with a strong critic, or regenerate (\dm).

\subsection{Judge and calibration assumptions}

\begin{assumption}[Oracle Slice: Simple Random Subsample]
\label{assum:oracle}
The labeled oracle slice is a simple random sample from the evaluation distribution, and labels $Y$ are unbiased measurements of the true outcome.
\end{assumption}

\paragraph{Plain English.} The prompts you label should be representative of the prompts you evaluate. Don't cherry-pick easy prompts or focus only on one domain.

\paragraph{When it matters.} All modes that use \autocal{} (\dm, \ips, \dr). Without this, calibration is biased.

\paragraph{How to check.} Compare the distribution of $S$ (and other covariates like prompt length, family) between the oracle slice and the full evaluation set. Use a two-sample test (e.g., Kolmogorov-Smirnov) or visual inspection.

\paragraph{What happens if it fails.} \autocal{} learns the wrong mapping, leading to biased $R = f(S)$. CIs are too narrow because they don't account for the selection bias. Fix: stratify labeling by prompt family or $S$ bins; use inverse-propensity weighting if the labeling mechanism is known.

\begin{assumption}[Judge Monotone Sufficiency (J2-M)]
\label{assum:j2m}
There exist monotone functions $h_Y$ and $h_W$ such that:
\begin{itemize}
\item $\E[Y \mid S] = h_Y(S)$ (outcome is monotone in score),
\item $\E[W \mid S] = h_W(S)$ (importance weight is monotone in score).
\end{itemize}
\end{assumption}

\paragraph{Plain English.} Higher judge scores should correspond (on average) to better outcomes, and the importance weight should vary smoothly with the score. This is the key to \autocal{} and \simcal.

\paragraph{When it matters.} All modes for \autocal{} (outcome); off-policy for \simcal{} (weights).

\paragraph{How to check.} 
\begin{itemize}
\item For $h_Y$: Bin the oracle slice by $S$ and check that mean $Y$ increases (or decreases) monotonically. Fit isotonic regression and inspect the fit.
\item For $h_W$: Bin logged data by $S$ and check that mean $W$ is monotone. Scatter plot of $W$ vs.\ $S$ should show a trend.
\end{itemize}

\paragraph{What happens if it fails.}
\begin{itemize}
\item If $\E[Y \mid S]$ is not monotone, \autocal{} can be badly miscalibrated. Use two-stage \autocal{} (spline index $\to$ isotonic) or add a coarse index (prompt family).
\item If $\E[W \mid S]$ is not monotone, \simcal{} may not help (or may hurt). Check the diagnostic; if monotonicity fails, consider model-based weight calibration or switch to \dr.
\end{itemize}

\begin{assumption}[Single-Index Monotone Sufficiency (J2-MX)]
\label{assump:j2mx}
There exists a one-dimensional index $T=g(S,X_{\mathrm{cov}})$ and a nondecreasing $\mu$ such that
$\E[Y\mid S,X_{\mathrm{cov}}]=\mu(T)$.
For off-policy weight calibration analogously assume
$\E[W\mid S,X_{\mathrm{cov}}]=\nu(T)$ with nondecreasing $\nu$.
\end{assumption}

\paragraph{Plain English.} When judge monotonicity fails at fixed $S$ due to slice effects (e.g., length or domain bias), we assume there's still a one-dimensional summary $T$ of $(S, X_{\mathrm{cov}})$ where monotonicity holds. This is weaker than J2-M but still preserves the core structural belief in rank-ordering.

\paragraph{When it matters.} All modes when using two-stage \autocal{} with covariates (\S\ref{sec:two-stage-autocal}). If plain J2-M holds, this is not needed.

\paragraph{How to check.} Bin the oracle slice by quantiles of $\widehat T$ and verify monotone $\bar Y$ across bins; for IPS/DR, verify monotone $\bar W$ across $\widehat T$ bins. Compare regional error before/after adding covariates.

\paragraph{What happens if it fails.} Add/adjust $X_{\mathrm{cov}}$ (e.g., response length, family), increase oracle labels in problematic regions, or fall back to plain monotone on $S$ with caveats.

\begin{assumption}[Judge Stability]
\label{assum:stability}
The judge's scoring function does not drift between the time the log is collected and the time fresh draws are evaluated. Formally: $s_t(X, A) = s_{t'}(X, A)$ for all $t, t'$ in the evaluation window.
\end{assumption}

\paragraph{Plain English.} The judge should assign the same score to the same $(X, A)$ pair regardless of when it's scored. Drift can come from model updates, prompt changes, or even randomness in sampling.

\paragraph{When it matters.} Off-policy (\ips, \dr) when comparing logged scores to fresh scores. Less critical for \dm{} if all scoring happens in one batch.

\paragraph{How to check.} Anchor set with Kendall $\tau$: score a fixed set of 20--50 prompts at $t_0$ and $t_1$, compute rank correlation. $\tau \ge 0.90$ is good; $\tau < 0.75$ is a red flag.

\paragraph{What happens if it fails.} Calibration breaks because the meaning of $S$ has changed. Oracle labels from $t_0$ don't predict outcomes at $t_1$. Fix: refresh the oracle slice with new labels from $t_1$ and re-fit \autocal. Or freeze the judge version/config during the evaluation window.

\begin{assumption}[Calibration Transportability]
\label{assum:transport}
Let $f$ be the judge$\to$oracle map learned on a source stratum $\mathcal{S}$ (e.g., a policy or time window).
We say $f$ is \emph{transportable} to a target stratum $\mathcal{T}$ if, for all $(S, X_{\mathrm{cov}})$ in the target support,
\[
\E\!\left[\,Y \,\middle|\, S, X_{\mathrm{cov}}, G=\mathcal{T}\right]
\;=\;
\E\!\left[\,Y \,\middle|\, S, X_{\mathrm{cov}}, G=\mathcal{S}\right]
\;=\; f^\star(S, X_{\mathrm{cov}})
\]
i.e., $Y \,\perp\, G \mid (S, X_{\mathrm{cov}})$, where $G$ is a group indicator (policy or time ``era'').
\end{assumption}

\paragraph{Interpretation.} Conditional on the score and any included covariates, the group (policy/time) carries no extra information about the outcome. Equivalently, $f$ is \emph{policy-invariant} and \emph{time-invariant} once you condition on $(S, X_{\mathrm{cov}})$.

\paragraph{Weaker variants.}
(i) \textbf{Mean-shift only:} $\E[Y\!\mid\!S,X_{\mathrm{cov}},G] = f^\star(S,X_{\mathrm{cov}}) + c_G$
(a constant vertical offset per group).
(ii) \textbf{Regional shift:} small bounded deviations within quantiles of the two-stage index $T=g(S,X_{\mathrm{cov}})$.

\paragraph{When it can fail.} If a new policy changes outcome at fixed score (judge bias w.r.t.\ that policy), or if the judge's semantics drift over time (model/rubric changes), $f$ is not transportable without adjustment.

\subsection{Structural assumptions}

\begin{assumption}[Bounded Outcomes]
\label{assum:bounded}
Outcomes $Y \in [0, 1]$ are bounded. Judge scores $S$ may be unbounded, but calibrated rewards $R = f(S) \in [0, 1]$.
\end{assumption}

\paragraph{Plain English.} The outcome you care about (pass rate, reward, satisfaction) has a natural scale. \autocal{} maps scores onto $[0, 1]$ to match that scale.

\paragraph{When it matters.} All modes. Boundedness is needed for mean preservation and honest CIs.

\paragraph{How to check.} Verify that oracle $Y \in [0, 1]$. If judge scores are on a different scale (e.g., 1--10), \autocal{} automatically rescales.

\paragraph{What happens if it fails.} If outcomes are unbounded (e.g., token count, latency), variance can be huge and isotonic regression may overfit. Consider transforming to a bounded scale (e.g., log-transform or winsorize) before calibration.

\subsection{Rate assumptions (for DR)}

\begin{assumption}[Critic and Weight Convergence Rates (R3)]
\label{assum:rates}
For doubly robust inference, either:
\begin{itemize}
\item The outcome model $\hat{g}(X)$ converges at rate $n^{-1/4}$ or faster, OR
\item The stabilized weights $W$ converge to the true weights at rate $n^{-1/4}$ or faster.
\end{itemize}
If both converge at $n^{-1/2}$, \dr{} achieves the efficiency bound.
\end{assumption}

\paragraph{Plain English.} For \dr{} to work well, you need either a decent critic or decent weights (or both). If both are terrible, \dr{} loses its advantages.

\paragraph{When it matters.} \dr{} only. Not needed for \dm{} or \ips.

\paragraph{How to check.} Orthogonality diagnostic. If the orthogonality score CI contains zero, you're in good shape. If not, either the critic or the weights are too far off.

\paragraph{What happens if it fails.} \dr{} is still consistent (unbiased) but loses efficiency and may have higher variance than \ips. The fix is to improve the critic (more fresh draws, better model) or improve the weights (better \simcal, cohort restriction).

\subsection{When assumptions can be relaxed}

\begin{itemize}
\item \textbf{Overlap (D2)} can be relaxed if you use \dr{} with a strong critic. The critic can extrapolate to regions with poor overlap.
\item \textbf{Monotone sufficiency (J2-M)} can be relaxed with two-stage \autocal{} or coarse indexing (prompt family).
\item \textbf{Judge stability (J3)} can be relaxed if you refresh the oracle slice frequently and re-calibrate.
\item \textbf{Rate assumptions (R3)} are not needed for point estimates, only for optimal inference. If you don't care about efficiency, you can ignore them.
\end{itemize}

\subsection{Sensitivity analysis and robustness checks}

When assumptions are questionable, perform sensitivity checks:

\begin{itemize}
\item \textbf{Cohort restriction:} Re-run the analysis on a subset with better overlap (e.g., $|\log W| < 2$). If estimates are stable, you're robust.
\item \textbf{Trimming:} Drop the top 1--5\% of weights and see if results change. Large changes indicate tail sensitivity.
\item \textbf{Alternative calibrators:} Compare isotonic \autocal{} to spline or linear calibration. Agreement suggests robustness.
\item \textbf{Alternative critics:} For \dr, try multiple outcome models (linear, random forest, boosting). If all agree, you're in good shape.
\item \textbf{Anchor set drift:} Compute $\tau$ on multiple anchor sets or at multiple time points. Consistent high $\tau$ indicates stability.
\end{itemize}

\subsection{Minimal assumptions by mode}

\begin{table}[h]
\centering
\begin{tabular}{lccc}
\toprule
Assumption & \dm & \ips & \dr \\
\midrule
SUTVA (D1) & \checkmark & \checkmark & \checkmark \\
Overlap (D2) & --- & \checkmark & \checkmark* \\
Oracle slice (J1) & \checkmark & \checkmark & \checkmark \\
Monotone sufficiency (J2-M) & \checkmark & \checkmark & \checkmark \\
Judge stability (J3) & \checkmark* & \checkmark & \checkmark \\
Bounded outcomes (S1) & \checkmark & \checkmark & \checkmark \\
Rate assumptions (R3) & --- & --- & \checkmark \\
\bottomrule
\end{tabular}
\caption{Assumptions required by mode. \checkmark* = required but less critical due to robustness. --- = not needed.}
\end{table}

\subsection{Summary}

The key assumptions are SUTVA, overlap (for off-policy), oracle randomness, and judge monotone sufficiency. Most are checkable via diagnostics. When in doubt, run sensitivity checks and report them alongside results. If assumptions fail and cannot be fixed, switch modes or regenerate data.

\section{Operator Playbook: Start to Finish}

\subsection{Overview}

This section is a step-by-step guide for practitioners. It covers the full evaluation workflow from data collection to reporting, with decision trees, checklists, and troubleshooting tips.

\subsection{Decision tree: which mode to use?}

\begin{enumerate}
\item \textbf{Can you generate fresh outputs for each candidate policy?}
   \begin{itemize}
   \item \textbf{Yes} $\to$ Can you afford to label an oracle slice (50--200 examples)?
      \begin{itemize}
      \item \textbf{Yes} $\to$ Use \textbf{\dm} (on-policy, simplest). If you also have logged data with logprobs, consider \textbf{\dr} for tighter CIs.
      \item \textbf{No} $\to$ Use \textbf{Direct mode without calibration} (raw judge scores, rank-only). No KPI-scale estimates, but still useful for ranking.
      \end{itemize}
   \item \textbf{No} $\to$ Do you have a judged log with logprobs for $\pi_0$ and candidates $\pi'$?
      \begin{itemize}
      \item \textbf{Yes} $\to$ Can you label an oracle slice?
         \begin{itemize}
         \item \textbf{Yes} $\to$ Use \textbf{\ips} or \textbf{\dr} (if you can generate a few fresh draws for a critic). Check ESS; if $< 10\%$, \dr{} is strongly recommended.
         \item \textbf{No} $\to$ Use \textbf{IPS without calibration} (relative comparisons only, no KPI scale).
         \end{itemize}
      \item \textbf{No} $\to$ You cannot perform causal off-policy evaluation. Regenerate or use \dm.
      \end{itemize}
   \end{itemize}
\end{enumerate}

\subsection{Workflow 1: Direct Modeling (DM)}

\paragraph{Step 1: Design the evaluation set.}
\begin{itemize}
\item Choose $m$ prompts representative of deployment (e.g., 500--2000).
\item Ensure diversity: cover prompt families, lengths, difficulty levels.
\item Use stratified sampling if you have subgroups of interest.
\end{itemize}

\paragraph{Step 2: Generate outputs.}
\begin{itemize}
\item For each policy $\pi$, generate one output per prompt.
\item Use the same random seed (or paired draws) across policies for variance reduction.
\item Log all outputs, prompts, and metadata (model version, temperature, etc.).
\end{itemize}

\paragraph{Step 3: Judge all outputs.}
\begin{itemize}
\item Apply the judge to all $(X_i, A_{\pi i})$ pairs to get scores $S_{\pi i}$.
\item Use a fixed judge config (version, prompt, temperature).
\item Log judge metadata (timestamps, version, config).
\end{itemize}

\paragraph{Step 4: Label an oracle slice.}
\begin{itemize}
\item Sample 50--200 examples uniformly at random from the evaluation set.
\item Obtain ground-truth labels $Y$ (human ratings, gold KPI, etc.).
\item Ensure labels are unbiased and representative.
\end{itemize}

\paragraph{Step 5: Fit AutoCal-R.}
\begin{itemize}
\item Use 5-fold cross-fitting to learn $f: S \to [0, 1]$.
\item Check diagnostics: coverage (aim for $\ge 95\%$), reliability (OOF MAE $< 0.05$), mean preservation ($|\bar{f(S)} - \bar{Y}| < 0.02$).
\item If regional error is high, enable two-stage \autocal{} or add a coarse index.
\end{itemize}

\paragraph{Step 6: Compute estimates and CIs.}
\begin{itemize}
\item Calibrate rewards: $R_{\pi i} = f(S_{\pi i})$.
\item Compute $\est{V}_{\dm}(\pi) = \frac{1}{m} \sum_i R_{\pi i}$.
\item For pairwise contrasts, compute $\est{\Delta}(\pi, \pi') = \frac{1}{m} \sum_i (R_{\pi i} - R_{\pi' i})$ (paired SE).
\item Add \oua{} by refitting $f$ across oracle folds and computing $\Var_{\oua}$.
\item Report 95\% CIs: $\est{V} \pm 1.96 \cdot \SE_{\text{total}}$.
\end{itemize}

\paragraph{Step 7: Report.}
\begin{itemize}
\item For each policy: $\est{V}$ with CI, \oua{} share, calibration diagnostics, coverage badge.
\item For key contrasts: $\est{\Delta}$ with paired CI.
\item Include diagnostic plots (calibration curve, coverage histogram).
\end{itemize}

\subsection{Workflow 2: Off-Policy IPS}

\paragraph{Step 1: Collect a judged log.}
\begin{itemize}
\item Deploy $\pi_0$ and log $(X_i, A_i, S_i)$ for $n$ prompts.
\item Store logprobs: $\log \pi_0(A_i \mid X_i)$ and $\log \pi'(A_i \mid X_i)$ for each candidate $\pi'$.
\end{itemize}

\paragraph{Step 2: Label an oracle slice.}
\begin{itemize}
\item Sample 50--200 examples uniformly from the logged data.
\item Obtain ground-truth $Y$.
\end{itemize}

\paragraph{Step 3: Fit AutoCal-R.}
\begin{itemize}
\item Same as DM: 5-fold cross-fitting, check diagnostics.
\end{itemize}

\paragraph{Step 4: Compute and stabilize weights.}
\begin{itemize}
\item Raw weights: $W_i = \exp(\log \pi'(A_i \mid X_i) - \log \pi_0(A_i \mid X_i))$.
\item Apply \simcal{} (5-fold, variance cap $\rho = 0.95$) to get $W_i^{\text{stab}}$.
\item Check ESS: $({\textstyle\sum} W^{\text{stab}})^2 / ({\textstyle\sum} (W^{\text{stab}})^2)$. Aim for $\ge 10\%$.
\item Check Hill index $\alpha \ge 2$.
\end{itemize}

\paragraph{Step 5: Compute estimates and CIs.}
\begin{itemize}
\item Calibrate rewards: $R_i = f(S_i)$.
\item Compute $\est{V}_{\ips}(\pi') = \frac{1}{n} \sum_i W_i^{\text{stab}} \cdot R_i$.
\item Compute SE via influence functions, add \oua, form 95\% CI.
\end{itemize}

\paragraph{Step 6: Report.}
\begin{itemize}
\item $\est{V}$ with CI, ESS (absolute and \%), Hill index, weight summary (min, median, max, 95th pct).
\item \oua{} share, calibration diagnostics, judge stability check (if available).
\end{itemize}

\subsection{Workflow 3: Off-Policy DR}

\paragraph{Steps 1--4: Same as IPS.}

\paragraph{Step 5: Generate fresh draws for the critic.}
\begin{itemize}
\item For each candidate $\pi'$, generate fresh outputs on the same prompts used in the log.
\item Judge these fresh outputs to get $S'_{\pi i}$ and calibrate to $R'_{\pi i} = f(S'_{\pi i})$.
\item This is your training set for the outcome model $\hat{g}(X)$.
\end{itemize}

\paragraph{Step 6: Train the critic (cross-fitted).}
\begin{itemize}
\item Split fresh draws into 5 folds.
\item For each fold $k$, train $\hat{g}_k(X)$ on the other 4 folds to predict $R'_{\pi}$.
\item Predict on fold $k$ to get $\hat{g}(X_i)$ for logged prompts $X_i$.
\item Use a flexible model (e.g., gradient-boosted trees, neural net) with regularization.
\end{itemize}

\paragraph{Step 7: Compute DR estimate.}
\begin{itemize}
\item $\est{V}_{\dr}(\pi') = \frac{1}{n} \sum_i \left[ W_i^{\text{stab}} \cdot (R_i - \hat{g}(X_i)) + \hat{g}(X_i) \right]$.
\item Compute orthogonality score: $\frac{1}{n} \sum_i (W_i^{\text{stab}} - 1) \cdot (R_i - \hat{g}(X_i))$ with CI.
\item If CI excludes zero, improve the critic or weights.
\end{itemize}

\paragraph{Step 8: Report.}
\begin{itemize}
\item $\est{V}$ with CI, ESS, Hill index, weight summary.
\item Orthogonality score with CI, critic OOF $R^2$.
\item \oua{} share, calibration diagnostics.
\end{itemize}

\subsection{Checklist: before you ship an estimate}

\begin{enumerate}
\item[$\square$] Checked score coverage ($\ge 85\%$)?
\item[$\square$] Checked calibration reliability (OOF MAE $< 0.10$, mean preservation OK)?
\item[$\square$] (Off-policy) Checked ESS ($\ge 10\%$) and Hill index ($\alpha \ge 2$)?
\item[$\square$] (DR) Checked orthogonality (CI contains zero)?
\item[$\square$] Checked \oua{} share (if $> 50\%$, consider adding labels)?
\item[$\square$] Checked judge stability (if using old logs)?
\item[$\square$] Reported CIs that include \oua?
\item[$\square$] Included diagnostic plots and summary in the report?
\item[$\square$] Documented judge config, prompt set, oracle sampling procedure?
\item[$\square$] Ran sensitivity checks (cohort restriction, trimming, alternative calibrators)?
\end{enumerate}

If any box is unchecked, do not ship the estimate.

\subsection{Troubleshooting guide}

\paragraph{Problem: Low ESS ($< 10\%$).}
\begin{itemize}
\item \textbf{Cause:} Poor overlap; policies are too different.
\item \textbf{Fixes:} (1) Apply \simcal{} with stricter $\rho$ (e.g., 0.90). (2) Restrict to high-overlap cohort ($|\log W| < 2$). (3) Switch to \dr{} with a strong critic. (4) If ESS $< 1\%$, regenerate and use \dm.
\end{itemize}

\paragraph{Problem: Heavy tails ($\alpha < 2$).}
\begin{itemize}
\item \textbf{Cause:} Extreme weights due to rare events or provider drift.
\item \textbf{Fixes:} (1) Use \simcal{} aggressively. (2) Cohort restriction. (3) Check for provider temperature/frequency-penalty changes. (4) Switch to \dr{} or regenerate.
\end{itemize}

\paragraph{Problem: Poor calibration (OOF MAE $> 0.10$).}
\begin{itemize}
\item \textbf{Cause:} Judge scores are not monotone in outcome, or regional miscalibration.
\item \textbf{Fixes:} (1) Enable two-stage \autocal. (2) Add a coarse index (prompt family, length). (3) Add 10--30 labels in problematic $S$ regions. (4) Check for judge drift.
\end{itemize}

\paragraph{Problem: Thin coverage ($< 85\%$).}
\begin{itemize}
\item \textbf{Cause:} Oracle slice does not span the evaluation $S$-range.
\item \textbf{Fixes:} (1) Add 5--20 labels targeting uncovered $S$ bins. (2) Narrow the prompt set to the intended deployment slice. (3) If stress-testing edges, ensure oracle includes edge examples.
\end{itemize}

\paragraph{Problem: DR orthogonality CI excludes zero.}
\begin{itemize}
\item \textbf{Cause:} Critic or weights are poor.
\item \textbf{Fixes:} (1) Improve critic: add regularization, use a more flexible model, increase fresh draw sample size. (2) Revisit \simcal{} or overlap diagnostics. (3) Check cross-fitting folds. (4) Fall back to \ips{} if orthogonality fails persistently.
\end{itemize}

\paragraph{Problem: Large \oua{} share ($> 50\%$).}
\begin{itemize}
\item \textbf{Cause:} Label scarcity is the bottleneck, not prompt scarcity.
\item \textbf{Fixes:} (1) Add labels, prioritizing sparse $S$ regions or high-variance prompt families. (2) Adding more prompts yields diminishing returns; focus labeling budget on coverage and balance.
\end{itemize}

\subsection{Sample size planning}

\paragraph{Goal:} Achieve a target CI half-width $\delta$ (e.g., $\delta = 0.02$ for $\pm 2\%$ precision).

\paragraph{Prompt sample size (for DM or DR with good critic):}
\begin{equation}
m \approx \left( \frac{1.96 \cdot \sigma_R}{\delta} \right)^2,
\end{equation}
where $\sigma_R$ is the SD of calibrated rewards $R$ (estimate from a pilot).

\paragraph{Label sample size (oracle):}
The \oua{} variance scales as $1 / n_{\text{oracle}}$. For \oua{} share $< 20\%$, use:
\begin{equation}
n_{\text{oracle}} \approx 5 \cdot m^{1/2}.
\end{equation}
For $m = 1000$, this gives $n_{\text{oracle}} \approx 150$.

\paragraph{Fresh draws for DR critic:}
Use the same $m$ prompts as the logged data. More is better; diminishing returns after $m \approx 500$.

\subsection{Reporting template}

\begin{lstlisting}
=== CJE Evaluation Report ===
Date: 2025-10-08
Evaluator: Alice
Mode: Calibrated DR
Estimator: stacked-dr

## Policies Evaluated
- Baseline (pi_0): gpt-3.5-turbo (deployed 2025-09-01 to 2025-10-01)
- Candidate (pi'): gpt-4-mini (under consideration)

## Data
- Logged prompts: n = 1,000
- Oracle slice: 150 (15% of total)
- Fresh draws (for critic): 1,000 (same prompts)
- Judge: GPT-4-as-judge with rubric v2.3 (frozen)
- Outcome: Binary pass rate (Y in {0, 1})

## Estimates
pi_0 (baseline): 0.72 [0.69, 0.75] (95% CI)
pi' (gpt-4-mini): 0.81 [0.78, 0.84] (95% CI)
Delta (pi' - pi_0): +0.09 [0.05, 0.13] (95% CI, p < 0.001)

## Diagnostics
[1] Score Coverage: 94% (PASS)
[2] Calibration Reliability: OOF MAE = 0.04 (PASS)
[3] ESS: 34% (WARN, acceptable for DR)
[4] Hill Index: alpha = 2.8 (PASS)
[5] DR Orthogonality: 0.03 [-0.02, 0.08] (PASS, CI contains 0)
[6] OUA Share: 15% (PASS)

## Interpretation
Switching from gpt-3.5-turbo to gpt-4-mini is estimated to increase
the pass rate by 9 percentage points (95% CI: [5%, 13%]). The estimate
is reliable: all diagnostics pass, ESS is moderate (acceptable for DR),
and orthogonality is good.

## Recommendation
Proceed with deployment of gpt-4-mini. Expected uplift is substantial
and statistically significant.

## Attachments
- calibration_curve.pdf
- weight_distribution.pdf
- orthogonality_residuals.pdf
- data_and_code.zip (for reproducibility)
\end{lstlisting}

\subsection{Summary}

The operator playbook provides decision trees, workflows, checklists, and troubleshooting for the full CJE pipeline. Always check diagnostics before shipping, document everything, and report CIs with \oua. When in doubt, run sensitivity checks.

\section{Case Studies}

\subsection{Overview}

This section presents three compact case studies illustrating CJE in practice: (1) a clean DM comparison, (2) an off-policy IPS evaluation with overlap challenges, and (3) a DR analysis combining logged data with fresh draws.

\subsection{Case 1: DM for model selection (on-policy)}

\paragraph{Setting.} A team is choosing between three candidate models for a code-generation task: GPT-3.5, GPT-4, and Claude-3. They have 1,000 diverse coding prompts and can generate one output per model per prompt. The KPI is ``correctness'' (binary: code passes unit tests or not). They use GPT-4-as-judge to score outputs on a 1--10 scale.

\paragraph{Data collection.}
\begin{itemize}
\item Generate outputs: 3 models $\times$ 1,000 prompts $=$ 3,000 responses.
\item Judge all 3,000 with GPT-4, get scores $S \in [1, 10]$.
\item Sample 150 responses uniformly, run unit tests to get ground truth $Y \in \{0, 1\}$.
\end{itemize}

\paragraph{Calibration.}
\begin{itemize}
\item Fit \autocal{} (5-fold isotonic regression) on the 150-sample oracle slice.
\item Diagnostics: Coverage 96\% (pass), OOF MAE 0.03 (pass), mean preservation $|\bar{f(S)} - \bar{Y}| = 0.01$ (excellent).
\item Map all 3,000 scores to calibrated rewards $R = f(S) \in [0, 1]$.
\end{itemize}

\paragraph{Estimates (with \oua).}
\begin{align*}
\est{V}_{\dm}(\text{GPT-3.5}) &= 0.62 \; [0.58, 0.66] \\
\est{V}_{\dm}(\text{GPT-4}) &= 0.78 \; [0.75, 0.81] \\
\est{V}_{\dm}(\text{Claude-3}) &= 0.81 \; [0.78, 0.84]
\end{align*}
Paired contrasts (tighter CIs due to pairing):
\begin{align*}
\est{\Delta}(\text{GPT-4}, \text{GPT-3.5}) &= +0.16 \; [0.11, 0.21] \\
\est{\Delta}(\text{Claude-3}, \text{GPT-4}) &= +0.03 \; [-0.01, 0.07]
\end{align*}

\paragraph{Diagnostics.}
\begin{itemize}
\item \oua{} share: 12\% (labels sufficient, variance dominated by prompt variation).
\item All diagnostics pass.
\end{itemize}

\paragraph{Conclusion.} Claude-3 and GPT-4 are statistically indistinguishable; both substantially outperform GPT-3.5. Team chooses Claude-3 for deployment based on cost.

\subsection{Case 2: IPS with overlap challenges (off-policy)}

\paragraph{Setting.} A team deployed GPT-3.5-turbo in production for a month, logging 10,000 prompts with responses, judge scores, and logprobs. They now want to estimate what the KPI would have been if they had deployed GPT-4-turbo instead, without regenerating.

\paragraph{Data.}
\begin{itemize}
\item Logged data: $n = 10{,}000$ with $(X_i, A_i, S_i, \log \pi_0(A_i \mid X_i), \log \pi'(A_i \mid X_i))$.
\item Oracle slice: 200 samples with ground truth $Y$ (user satisfaction, binary).
\item No fresh draws (cannot generate for GPT-4-turbo on these old prompts).
\end{itemize}

\paragraph{Calibration.}
\begin{itemize}
\item Fit \autocal{} on 200-sample oracle.
\item Diagnostics: Coverage 89\% (warn), OOF MAE 0.06 (pass), mean preservation OK.
\item Action: Add 20 labels targeting low-$S$ bins to improve coverage to 94\%.
\end{itemize}

\paragraph{Weights and stabilization.}
\begin{itemize}
\item Raw weights: $W_i = \exp(\log \pi'(A_i \mid X_i) - \log \pi_0(A_i \mid X_i))$.
\item Raw ESS: 8\% (poor overlap; GPT-4-turbo is quite different from GPT-3.5-turbo).
\item Apply \simcal{} with $\rho = 0.95$: stabilized ESS improves to 18\%.
\item Hill index $\alpha = 2.6$ (pass, finite variance).
\end{itemize}

\paragraph{Estimate.}
\begin{align*}
\est{V}_{\ips}(\text{GPT-4-turbo}) &= 0.74 \; [0.68, 0.80] \\
\text{(Baseline GPT-3.5-turbo)} &= 0.65 \; [0.62, 0.68] \\
\est{\Delta} &= +0.09 \; [0.02, 0.16]
\end{align*}

\paragraph{Diagnostics.}
\begin{itemize}
\item ESS 18\% (warn, but acceptable given no fresh draws available).
\item \oua{} share 22\% (warn; labels are becoming a bottleneck).
\item Weight max/median ratio: 45 (moderate tail).
\end{itemize}

\paragraph{Sensitivity check.}
\begin{itemize}
\item Cohort restriction: Re-run on prompts with $|\log W| < 2$ ($n = 6{,}000$). ESS improves to 35\%, estimate is $0.76 \; [0.71, 0.81]$ (consistent).
\item Trimming: Drop top 1\% of weights. Estimate shifts to $0.73 \; [0.68, 0.78]$ (minor change, robust).
\end{itemize}

\paragraph{Conclusion.} Switching to GPT-4-turbo would likely improve satisfaction by $\approx 9$ percentage points. The estimate is moderately reliable (ESS 18\%, sensitivity checks agree). Team decides to run a small A/B test to confirm before full deployment.

\subsection{Case 3: DR for tight CIs (off-policy + fresh draws)}

\paragraph{Setting.} Same as Case 2, but the team can now afford to generate 2,000 fresh responses from GPT-4-turbo (20\% of the logged data) to train a critic for DR.

\paragraph{Data.}
\begin{itemize}
\item Logged data: $n = 10{,}000$ (same as Case 2).
\item Oracle slice: 200 (same).
\item Fresh draws: 2,000 prompts sampled uniformly from the 10k, with GPT-4-turbo outputs and judge scores.
\end{itemize}

\paragraph{Weights and calibration.} Same as Case 2 (ESS 18\% after \simcal, $\alpha = 2.6$).

\paragraph{Critic training.}
\begin{itemize}
\item Train a gradient-boosted tree $\hat{g}(X)$ on the 2,000 fresh draws (5-fold cross-fitted) to predict calibrated reward $R$.
\item OOF $R^2 = 0.58$ (moderate; prompt features include length, first-token embedding, topic cluster).
\end{itemize}

\paragraph{DR estimate.}
\begin{align*}
\est{V}_{\dr}(\text{GPT-4-turbo}) &= 0.75 \; [0.71, 0.79]
\end{align*}

\paragraph{Diagnostics.}
\begin{itemize}
\item Orthogonality score: $0.02 \; [-0.03, 0.07]$ (pass, CI contains zero).
\item ESS: still 18\% (unchanged by DR; weights are the same).
\item \oua{} share: 18\% (slightly lower than IPS due to variance reduction from the critic).
\item Compare to IPS: IPS gave $[0.68, 0.80]$ (width 0.12); DR gives $[0.71, 0.79]$ (width 0.08). DR is 33\% tighter.
\end{itemize}

\paragraph{Conclusion.} DR delivers a tighter CI and a more stable estimate. The point estimate (0.75) is consistent with IPS (0.74) and the sensitivity checks. The team is now confident enough to deploy GPT-4-turbo without an A/B test.

\subsection{Lessons learned}

\begin{itemize}
\item \textbf{DM is simplest when you can generate.} Case 1 shows clean paired comparisons with tight CIs.
\item \textbf{IPS works when overlap is moderate.} Case 2 shows that ESS 18\% is usable, especially with sensitivity checks.
\item \textbf{DR tightens CIs and boosts confidence.} Case 3 shows a 33\% CI reduction with only 2,000 fresh draws (20\% of logged data).
\item \textbf{Always check diagnostics.} Coverage, ESS, orthogonality, and \oua{} share catch issues early.
\item \textbf{Sensitivity checks build trust.} Cohort restriction, trimming, and alternative calibrators confirm robustness.
\end{itemize}

\section{Implementation Notes}

\subsection{Overview}

This section covers practical implementation details: data formats, software requirements, computational considerations, and integration with existing evaluation pipelines.

\subsection{Data format and schema}

CJE expects data in JSONL format (one JSON object per line). Each record represents one sample (prompt-response-score triple).

\paragraph{Minimal schema (for DM):}
\begin{lstlisting}[language=Python,caption=DM Data Schema]
{
  "prompt_id": "prompt_001",
  "prompt": "Write a function to reverse a string",
  "response": "def reverse(s): return s[::-1]",
  "policy": "gpt-4",
  "metadata": {
    "judge_score": 8.5,
    "oracle_label": 1  # optional, only for oracle slice
  }
}
\end{lstlisting}

\paragraph{Extended schema (for IPS/DR):}
\begin{lstlisting}[language=Python,caption=Off-Policy Data Schema]
{
  "prompt_id": "prompt_001",
  "prompt": "Write a function to reverse a string",
  "response": "def reverse(s): return s[::-1]",
  "base_policy": "gpt-3.5-turbo",
  "base_policy_logprob": -12.34,
  "target_policy_logprobs": {
    "gpt-4": -10.56,
    "claude-3": -11.23
  },
  "metadata": {
    "judge_score": 8.5,
    "oracle_label": 1,  # optional
    "timestamp": "2025-10-01T12:00:00Z",
    "prompt_family": "code_generation",
    "response_length": 123,     # auto-computed if enabled
    "domain": "math",           # user-provided
    "difficulty": "hard"
  }
}
\end{lstlisting}

\paragraph{Schema note: covariates.} Place covariates in \texttt{metadata}. CJE can auto-compute \texttt{response\_length} from the \texttt{response} field if \texttt{include\_response\_length=True}. Other covariates (e.g., domain, difficulty) must be provided manually. Covariates are used in two-stage \autocal{} for regional adaptation (\S\ref{sec:two-stage-autocal}) and in DR outcome models for better predictions.

\paragraph{Fresh draws (for DR):}
\begin{lstlisting}[language=Python,caption=Fresh Draw Schema]
{
  "prompt_id": "prompt_001",
  "prompt": "Write a function to reverse a string",
  "response": "def reverse(s): return s[::-1]",
  "policy": "gpt-4",
  "draw_type": "fresh",
  "metadata": {
    "judge_score": 9.0
  }
}
\end{lstlisting}

\subsection{Software requirements}

CJE is implemented in Python and requires:
\begin{itemize}
\item Python 3.9+
\item NumPy, SciPy (for numerical computation)
\item scikit-learn (for isotonic regression, cross-validation)
\item Pandas (for data manipulation)
\item Matplotlib, Seaborn (for visualization)
\item statsmodels (for influence functions and inference)
\end{itemize}

Optional:
\begin{itemize}
\item XGBoost or LightGBM (for DR critic training)
\item Jupyter (for interactive analysis)
\end{itemize}

\subsection{Installation and quickstart}

\begin{lstlisting}[language=bash,caption=Installation]
# Install from PyPI
pip install cje-eval

# Or install from source
git clone https://github.com/cimo-labs/cje.git
cd cje
pip install -e .
\end{lstlisting}

\begin{lstlisting}[language=Python,caption=Quickstart Example]
from cje import analyze_dataset

# Load your data (JSONL format)
results = analyze_dataset(
    logged_data_path="evaluation_data.jsonl",
    fresh_draws_dir="responses/",         # Optional: for DR mode
    calibration_data_path="oracle_labels.jsonl",  # Optional: dedicated calibration set
    timestamp_field="timestamp",          # Optional: for drift detection
    check_drift=True,                     # Optional: automated drift detection
    estimator="auto",  # auto-selects DM, IPS, or DR
    judge_field="judge_score",
    oracle_field="oracle_label",
    verbose=True
)

# Inspect results
print(results.summary())
results.plot_diagnostics()

# Check for drift (if enabled)
if "drift_diagnostics" in results.metadata:
    drift = results.metadata["drift_diagnostics"]
    if drift["drift_detection"]["has_drift"]:
        print("Warning: Judge drift detected")
\end{lstlisting}

\paragraph{Enabling two-stage with covariates (code).}
\begin{lstlisting}[language=Python]
results = analyze_dataset(
    logged_data_path="logs.jsonl",
    fresh_draws_dir="responses/",       # optional (DR)
    calibration_mode="auto",            # or "two_stage"
    include_response_length=True,       # derives from response text
    calibration_covariates=["domain","difficulty"]  # must exist in metadata
)
\end{lstlisting}

\subsection{Computational complexity}

\paragraph{Calibration (AutoCal-R):}
\begin{itemize}
\item Isotonic regression: $O(n_{\text{oracle}} \log n_{\text{oracle}})$ per fold.
\item Cross-fitting (5 folds): $O(K \cdot n_{\text{oracle}} \log n_{\text{oracle}})$.
\item Typical: 150 oracle samples, 5 folds $\to$ $< 1$ second.
\end{itemize}

\paragraph{Weight calibration (SIMCal):}
\begin{itemize}
\item Isotonic regression for $W$ vs.\ $S$: $O(n \log n)$ per candidate.
\item Stacking (3 candidates, 5 folds): $O(K \cdot n \log n)$.
\item Typical: 10,000 samples, 5 folds $\to$ $\approx 5$ seconds.
\end{itemize}

\paragraph{DR critic training:}
\begin{itemize}
\item Depends on model choice. Gradient-boosted trees: $O(m \cdot d \cdot \text{n\_trees})$, where $m$ is fresh draw sample size, $d$ is feature dimension.
\item Typical: 2,000 fresh draws, 100 trees, 50 features $\to$ $\approx 30$ seconds.
\end{itemize}

\paragraph{Influence functions and inference:}
\begin{itemize}
\item $O(n)$ for IPS; $O(n + m)$ for DR (logged + fresh).
\item Typical: 10,000 samples $\to$ $< 1$ second.
\end{itemize}

\paragraph{Total runtime:} For a typical off-policy DR analysis (10k logged, 2k fresh, 150 oracle), expect $< 1$ minute on a laptop.

\subsection{Parallelization and scaling}

\begin{itemize}
\item \textbf{Cross-fitting:} Folds are independent; parallelize across folds (5-10x speedup).
\item \textbf{Multiple policies:} Evaluate policies in parallel (one per core).
\item \textbf{Large datasets:} For $n > 100{,}000$, use mini-batch stacking or subsample for weight calibration (weights are i.i.d. after calibration, so subsampling is safe).
\item \textbf{Distributed:} For $n > 1{,}000{,}000$, shard by prompt and aggregate influence functions.
\end{itemize}

\subsection{Integration with existing pipelines}

\paragraph{A/B testing platforms:} CJE complements A/B tests. Use CJE for rapid offline iteration; validate winners with online A/B tests before full deployment.

\paragraph{LLM-as-judge frameworks:} CJE is agnostic to the judge. Plug in any scoring function (OpenAI moderation API, custom rubric, ensemble of judges). Just ensure stability.

\paragraph{Labeling workflows:} Integrate with labeling platforms (e.g., Scale AI, Labelbox). Export a stratified sample for labeling, re-import labels, and run CJE.

\paragraph{CI/CD:} Embed CJE in CI pipelines. Run nightly evaluations on held-out test sets, flag regressions, and auto-generate diagnostic reports.

\subsection{Advanced features}

\paragraph{Stratified evaluation:} Evaluate subgroups (e.g., by prompt family, user segment) separately. CJE supports stratified analysis with per-stratum diagnostics.

\paragraph{Covariate adjustment:} Add prompt-level covariates (length, topic, difficulty) to the critic for better predictions.

\paragraph{Multi-outcome evaluation:} Evaluate multiple KPIs simultaneously (e.g., correctness, helpfulness, safety). Fit separate calibrators for each outcome.

\paragraph{Sequential testing:} For online settings, CJE can be adapted for sequential A/B testing with always-valid CIs (requires additional assumptions; see advanced docs).

\subsection{Debugging and diagnostics}

\paragraph{Common issues:}
\begin{itemize}
\item \textbf{``ESS too low'' error:} Check overlap; try cohort restriction or switch to DR.
\item \textbf{``Coverage < 50\%'' warning:} Add labels in uncovered $S$ bins or narrow prompt set.
\item \textbf{NaN estimates:} Likely a refusal gate triggered (ESS $< 1\%$ or $\alpha < 2$). Inspect \texttt{diagnostics.summary()}.
\item \textbf{Import errors:} Ensure \texttt{pip install -e .} was run in the CJE root directory.
\end{itemize}

\paragraph{Verbose mode:}
\begin{lstlisting}[language=Python]
results = analyze_dataset(..., verbose=True)
# Prints step-by-step progress and intermediate diagnostics
\end{lstlisting}

\paragraph{Diagnostic export:}
\begin{lstlisting}[language=Python]
results.export_diagnostics("diagnostics/")
# Saves all plots, tables, and metadata for audit
\end{lstlisting}

\subsection{Summary}

CJE is designed for ease of integration: JSONL data format, simple API, fast runtime ($< 1$ min for typical use cases), and rich diagnostics. Advanced users can extend with stratification, multi-outcome evaluation, and distributed scaling.

\section{Limitations and Known Issues}

\subsection{Overview}

CJE is a practical framework for causal judge evaluation, but it has limitations. This section documents known issues, settings where CJE may not apply, and open challenges.

\subsection{Limitation 1: Reliance on judge quality}

\paragraph{The issue.} CJE calibrates judge scores to outcome-scale rewards, but if the judge is systematically biased or unreliable, calibration cannot fix it. Garbage in, garbage out.

\paragraph{When it matters.} All modes. If the judge is poorly correlated with the true outcome ($\rho < 0.5$), even with perfect calibration, estimates will be noisy or biased.

\paragraph{Mitigation.}
\begin{itemize}
\item Use a high-quality judge (e.g., GPT-4, Claude-3 with a well-designed rubric).
\item Validate judge-human agreement on a small pilot (aim for $\rho > 0.7$).
\item Use ensemble judges (average scores from multiple judges) to reduce variance.
\item Check calibration reliability: if OOF MAE $> 0.10$, the judge may not be sufficiently predictive.
\end{itemize}

\subsection{Limitation 2: Monotonicity assumption (J2-M)}

\paragraph{The issue.} \autocal{} and \simcal{} assume that outcomes and weights are monotone in the judge score $S$. If $\E[Y \mid S]$ is non-monotone (e.g., $U$-shaped), isotonic regression can be badly miscalibrated.

\paragraph{When it matters.} All modes for \autocal; off-policy for \simcal.

\paragraph{Mitigation.}
\begin{itemize}
\item Check monotonicity on the oracle slice: bin by $S$, plot mean $Y$.
\item If non-monotone, use two-stage \autocal{} (flexible index $\to$ isotonic) or add a coarse index (prompt family, length).
\item For \simcal, check $W$ vs.\ $S$ scatter; if clearly non-monotone, consider model-based weight calibration (e.g., fit $\E[W \mid S]$ with a flexible model).
\end{itemize}

\subsection{Limitation 3: Overlap (off-policy only)}

\paragraph{The issue.} Off-policy evaluation requires that the logging policy $\pi_0$ could have generated any response the target policy $\pi'$ might generate. If overlap is poor, importance weights explode and estimates are unreliable.

\paragraph{When it matters.} Off-policy only (\ips, \dr). Not an issue for \dm.

\paragraph{Mitigation.}
\begin{itemize}
\item Check ESS; if $< 10\%$, overlap is poor.
\item Use \simcal{} to stabilize weights.
\item Restrict to high-overlap cohorts.
\item Switch to \dr{} with a strong critic (critic can extrapolate to low-overlap regions).
\item If overlap is catastrophic (ESS $< 1\%$), regenerate and use \dm.
\end{itemize}

\subsection{Limitation 4: Judge drift}

\paragraph{The issue.} If the judge's interpretation of scores changes over time (due to model updates, prompt changes, or even sampling randomness), calibration breaks. Old oracle labels don't predict new outcomes.

\paragraph{When it matters.} Off-policy when comparing old logs to new fresh draws; also \dm{} if scoring happens over a long time window.

\paragraph{Mitigation.}
\begin{itemize}
\item Freeze judge version and config during the evaluation window.
\item Check stability via an anchor set (Kendall $\tau \ge 0.90$).
\item If drift is detected, refresh the oracle slice with new labels and re-fit \autocal.
\item For long-running evaluations, re-calibrate periodically (e.g., monthly).
\end{itemize}

\subsection{Limitation 5: Small oracle slices}

\paragraph{The issue.} When the oracle slice is small ($< 50$ samples), \autocal{} has high variance, leading to large \oua{} share and wide CIs. Calibration may also be unreliable (overfitting).

\paragraph{When it matters.} All modes. Especially problematic when the evaluation $S$-range is wide or when regional calibration is needed.

\paragraph{Mitigation.}
\begin{itemize}
\item Aim for $n_{\text{oracle}} \ge 100$ (150--200 recommended).
\item Check \oua{} share; if $> 50\%$, add labels.
\item Prioritize labels in sparse $S$ regions or high-variance prompt families.
\item Use regularization (smoothed isotonic regression) to reduce overfitting on small oracle slices.
\end{itemize}

\subsection{Limitation 6: Extrapolation beyond labeled range}

\paragraph{The issue.} If evaluation scores $S$ fall outside the oracle $S$-range, \autocal{} must extrapolate. Extrapolation is unreliable, especially at the boundaries.

\paragraph{When it matters.} All modes. Coverage diagnostic flags this.

\paragraph{Mitigation.}
\begin{itemize}
\item Add labels targeting uncovered $S$ bins.
\item Narrow the prompt set to the intended deployment slice (avoid stress-testing edges unless you label them).
\item Inspect boundary slopes; if steep or flat, extrapolation is risky.
\item Report coverage badge; if $< 85\%$, flag the estimate as provisional.
\end{itemize}

\subsection{Limitation 7: High-dimensional prompts (DR critic)}

\paragraph{The issue.} For \dr, the critic $\hat{g}(X)$ must predict calibrated rewards from prompt features. If prompts are high-dimensional or unstructured (e.g., raw text), the critic may overfit or fail to generalize.

\paragraph{When it matters.} \dr{} only.

\paragraph{Mitigation.}
\begin{itemize}
\item Use feature engineering: extract prompt length, topic embeddings, difficulty scores.
\item Use flexible models with regularization (gradient-boosted trees, neural nets with dropout).
\item Check orthogonality: if CI excludes zero, the critic is not good enough; add features or fresh draws.
\item For very high-dimensional prompts, consider dimensionality reduction (e.g., PCA on embeddings).
\end{itemize}

\subsection{Limitation 8: Non-i.i.d. data}

\paragraph{The issue.} CJE assumes samples are independent and identically distributed (i.i.d.). If prompts are sequential or correlated (e.g., multi-turn conversations), SUTVA fails and estimates are biased.

\paragraph{When it matters.} All modes. Especially problematic in online settings with user sessions.

\paragraph{Mitigation.}
\begin{itemize}
\item For batch offline evaluation, ensure prompts are sampled independently.
\item For sequential data, break into independent chunks or use session-level aggregation.
\item If SUTVA is violated, redefine the estimand to account for interference (beyond this playbook).
\end{itemize}

\subsection{Limitation 9: Multiple testing}

\paragraph{The issue.} If you evaluate many policies or run many subgroup analyses, some will appear significant by chance (false positives). Standard CIs do not account for multiple comparisons.

\paragraph{When it matters.} When evaluating $> 5$ policies or running many subgroup analyses.

\paragraph{Mitigation.}
\begin{itemize}
\item Use Bonferroni or Benjamini-Hochberg correction for multiple testing.
\item Pre-specify a small number of key comparisons; treat others as exploratory.
\item Use a holdout set for final validation (train/eval/test split).
\end{itemize}

\subsection{Limitation 10: Generalization to deployment}

\paragraph{The issue.} CJE estimates are valid for the evaluation prompt distribution. If deployment prompts differ materially, estimates may not generalize.

\paragraph{When it matters.} All modes. This is a distribution-shift problem, not a CJE limitation per se.

\paragraph{Mitigation.}
\begin{itemize}
\item Ensure evaluation prompts are representative of deployment (stratified sampling by user segment, use case, etc.).
\item For new domains, collect a small deployment-representative slice and re-calibrate.
\item Monitor online KPIs after deployment and compare to offline estimates (meta-learning for future calibrations).
\end{itemize}

\subsection{Open challenges}

\begin{itemize}
\item \textbf{Sequential and adaptive evaluation:} How to handle multi-turn conversations or adaptive prompting where responses depend on prior context.
\item \textbf{Distribution shift:} Principled methods for extrapolating to new prompt distributions.
\item \textbf{Multi-objective optimization:} Evaluating trade-offs between multiple KPIs (e.g., correctness vs.\ safety vs.\ latency).
\item \textbf{Online learning:} Integrating CJE with bandit algorithms for continuous policy improvement.
\item \textbf{Fairness and robustness:} Ensuring estimates are unbiased across demographic groups and robust to adversarial prompts.
\end{itemize}

\subsection{Summary}

CJE's main limitations are judge quality, monotonicity assumptions, overlap (for off-policy), and oracle slice size. Most are checkable via diagnostics and addressable with targeted fixes. When limitations cannot be resolved, report them transparently and interpret estimates with appropriate caution.

\section{Conclusion}

\subsection{Summary}

Causal Judge Evaluation (CJE) provides a principled, practitioner-first framework for turning LLM-as-judge scores into causally interpretable estimates with honest confidence intervals. This playbook has covered:

\begin{itemize}
\item \textbf{The problem:} Naive averaging of judge scores is a heuristic that breaks in predictable ways (wrong scale, hidden drift, no uncertainty, off-policy illusion).

\item \textbf{The solution:} Three analysis modes---\dm{} (on-policy), Calibrated \ips{} (off-policy with weights), and Calibrated \dr{} (off-policy with critic)---each tailored to common evaluation workflows.

\item \textbf{The core methods:} \autocal{} to map scores to outcomes, \simcal{} to stabilize weights, and \oua{} to keep CIs honest.

\item \textbf{The diagnostics:} Five high-leverage checks (coverage, reliability, ESS, tail index, orthogonality) that catch the most important failure modes and point to concrete fixes.

\item \textbf{The assumptions:} SUTVA, overlap, oracle randomness, and judge monotone sufficiency, all checkable via diagnostics.

\item \textbf{The playbook:} Decision trees, workflows, checklists, troubleshooting, and reporting templates for operators.

\item \textbf{The evidence:} Case studies showing DM, IPS, and DR in action, with realistic numbers and sensitivity checks.

\item \textbf{The limitations:} Judge quality, monotonicity, overlap, drift, and small oracle slices are the main constraints; most are addressable with targeted remediations.
\end{itemize}

\subsection{When to use CJE}

Use CJE when you need to:
\begin{itemize}
\item Rank candidate policies on a shared prompt set with statistical rigor.
\item Estimate policy values on a meaningful KPI scale (e.g., pass rate, satisfaction) with confidence intervals.
\item Reuse judged logs to assess multiple candidates without regenerating (off-policy evaluation).
\item Combine logged data with fresh draws for tighter CIs and robustness (\dr).
\item Audit and document evaluation assumptions, diagnostics, and sensitivity checks.
\end{itemize}

\subsection{What CJE is not}

CJE is not:
\begin{itemize}
\item A replacement for A/B testing. Use CJE for rapid offline iteration; validate winners with online tests when stakes are high.
\item A fix for poor judges. If the judge is biased or unreliable, calibration cannot save you.
\item A method for sequential or adaptive settings (multi-turn conversations with context dependence). Extensions are possible but beyond this playbook.
\item A universal solution. When assumptions fail and cannot be fixed, CJE may not apply. In such cases, report limitations transparently.
\end{itemize}

\subsection{Best practices}

\begin{enumerate}
\item \textbf{Design for paired comparisons.} Use the same prompts and seeds across policies to reduce variance.

\item \textbf{Invest in oracle quality.} A well-chosen, representative oracle slice (150--200 samples) is the foundation of reliable calibration.

\item \textbf{Check diagnostics before shipping.} Coverage, ESS, reliability, and \oua{} share catch most issues. Always inspect them.

\item \textbf{Report CIs with OUA.} Honest intervals that include calibrator uncertainty build trust and prevent overconfidence.

\item \textbf{Run sensitivity checks.} Cohort restriction, trimming, and alternative calibrators confirm robustness and flag fragility.

\item \textbf{Document everything.} Judge config, prompt sampling, oracle slice, estimator choice, and diagnostic results. Reproducibility matters.

\item \textbf{Iterate.} Use CJE diagnostics to guide data collection: add labels where coverage is thin, improve the judge where reliability is poor, collect fresh draws where ESS is low.
\end{enumerate}

\subsection{Future directions}

Ongoing and future work includes:
\begin{itemize}
\item \textbf{Sequential and adaptive evaluation:} Extensions for multi-turn conversations and adaptive prompting.
\item \textbf{Multi-objective optimization:} Methods for evaluating trade-offs across multiple KPIs.
\item \textbf{Online integration:} Combining CJE with bandit algorithms for continuous policy improvement.
\item \textbf{Fairness and robustness:} Ensuring unbiased estimates across demographic groups and robust to adversarial prompts.
\item \textbf{Meta-learning for calibration:} Using historical evaluations to improve calibration on new domains.
\end{itemize}

\subsection{Acknowledgments}

CJE builds on decades of work in causal inference, importance sampling, doubly robust estimation, and calibration. Key intellectual debts include:
\begin{itemize}
\item Doubly robust methods (Robins, Rotnitzky, Scharfstein; Bang, Robins; Chernozhukov et al.).
\item Importance sampling and overlap diagnostics (Hirano, Imbens, Ridder; Li, Sävje, Sussman).
\item Isotonic regression and calibration (Zadrozny, Elkan; Platt; Guo et al.).
\item Influence functions and semiparametric efficiency (van der Laan, Robins).
\end{itemize}

This playbook synthesizes these methods into a practitioner-first toolkit for LLM evaluation.

\subsection{Final thoughts}

LLM-as-judge evaluation is here to stay. As models improve, so do the stakes: small differences in policy performance can translate to large impacts at scale. CJE provides a rigorous, transparent, and practical framework for making these decisions with confidence.

The diagnostics are not gatekeepers---they are guides. When diagnostics pass, trust the estimate. When they fail, use the fixes. When fixes don't work, report the limitations and proceed with caution (or regenerate).

Evaluation is not a one-shot activity. It's a loop: design, collect, calibrate, diagnose, remediate, report, and iterate. CJE is designed to support that loop with minimal friction and maximum transparency.

We hope this playbook serves as a practical companion for teams building and deploying LLMs. For questions, contributions, or feedback, visit the CJE repository at \url{https://github.com/your-org/cje}.

\vspace{1em}
\noindent\textbf{Happy evaluating!}


% Bibliography
\bibliographystyle{plain}
\bibliography{references}

% Footer note on last page
\vfill
\begin{center}
\small
\textit{This document was auto-generated from the CJE repository at tag v\cjeversion} \\
\url{https://github.com/cimo-labs/cje/tree/v\cjeversion/docs/playbook}
\end{center}

\end{document}
